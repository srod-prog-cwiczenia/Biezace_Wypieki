%%%%%%%%%%%%%%%%%%%%%%%%%%%%%%%%%%%%%%%%%%%%%%%%%%%%%%%%%%%%%%%%%%%%%%%%%
%                                                                       %
%                  praca o ideale (a)                                   %
%                                                                       %
%%%%%%%%%%%%%%%%%%%%%%%%%%%%%%%%%%%%%%%%%%%%%%%%%%%%%%%%%%%%%%%%%%%%%%%%%
%\documentclass[12pt]{article}
%%% aby uzyc amsart w pelnej krasie:
% 1. odkomentowac \address i \email
% 2. odkomentowac \keywords{} \subjclass
\documentclass[12pt]{amsart}
\usepackage{amssymb}
\usepackage{amsmath}
\usepackage{latexsym}
\usepackage{amsfonts}
\usepackage{enumerate}
\usepackage{mathtools}
%\usepackage[mathscr]{eucal}
\usepackage[mathscr]{euscript}
\usepackage{eqlist}
\usepackage{amsthm}
%%%%%%%%%%%%%%% Polish letter packages %%%%%%%%%%%%%%%%%%%%%%%%%%%%%%%%
%\usepackage[polish]{babel}
%\usepackage[utf8]{inputenc}
%\usepackage{t1enc}
%%% pakiet do generowania belkotu (wypelniacza):
\usepackage{lipsum}
%---------------------------------------------------------------------
% Theorems 
%---------------------------------------------------------------------
 
% Theorem style 'plain' are for: Theorem, Lemma, Corollary,
% Proposition, Conjecture, Criterion, Algorithm
\theoremstyle{plain}
\newtheorem{theorem}{Theorem}[section]
\newtheorem{lemma}[theorem]{Lemma}
\newtheorem{corollary}[theorem]{Corollary}
\newtheorem{conclusion}[theorem]{Corollary}
\newtheorem{claim}[theorem]{Claim}
\newtheorem{fact}[theorem]{Fact}
\newtheorem{proposition}[theorem]{Proposition}
\newtheorem{axiom}{Axiom}
% Theorem style 'definition' are for: Definition, Condition, Problem,
% Example
\theoremstyle{definition}
\newtheorem{definition}[theorem]{Definition}
\newtheorem{example}[theorem]{Example}
\newtheorem{exercise}{Exercise}
\newtheorem*{solution}{Solution}
\newtheorem{remark}[theorem]{Remark}
\newtheorem{Problem}[theorem]{Problem}
% Theorem style 'remark' are for: Remark, Note, Notation, Claim,
% Summary, Acknowledgement, Case, Conclusion
\theoremstyle{remark}
\newtheorem*{notation}{Notation}
\newtheorem*{acknowledgement}{Acknowledgement}

%%%%%% makra:

%A
\newcommand{\afc}{AFC}
\newcommand{\afcbar}{\overline{AFC}}
\newcommand{\arr}{\rightarrow}
\newcommand{\Arr}{\Rightarrow}

%B
\newcommand{\baire}{\omega^{\omega}}
\newcommand{\Baire}{\mathfrak{Baire}}
\newcommand{\Bor}{\mbox{${\mathcal B}or$}}
%Previous seems to be much finer than next.
%\newcommand{\Bor}{{\it Bor}}
\newcommand{\borelucrz}{Borel-UCR_0}

%C
\newcommand{\ca}{2^{\omega}}
\newcommand{\cantor}{\ca}
\newcommand{\Card}[1]{\Vert #1 \Vert}
\newcommand{\cl}{\mathit{cl}}

%D
\newcommand{\dom}{{\rm dom}}
\newcommand{\dummy}{{\tt Blah blah blah}}

%E
\newcommand{\Even}{\hbox{\rm \tiny Even}}

%F
\newcommand{\finsub}{[\omega]^{<\omega}}
\newcommand{\forces}{\mathrel{\|}\joinrel\mathrel{-}}

%G
\newcommand{\Graph}{\hbox{\it Graph}}

%H
\newcommand{\homeomorphic}{\approx}

%I
\newcommand{\incr}{\omega^{\uparrow \omega }}
\newcommand{\infsub}{[\omega]^{\omega}}

%L
\newcommand{\la}{\langle}

%M
\newcommand{\meager}{{\mathcal{MGR}}}
\newcommand{\minideal}{${\cal F}_{\hbox{\rm \scriptsize min}}(\neg
  D)\;$}

%N
\newcommand{\neglig}{{\cal N}}
\newcommand{\nnatural}{\mathbb{N}}

%O
\newcommand{\Odd}{\hbox{\rm \tiny Odd}}

%P
\newcommand{\Part}{{\it Part}}
\newcommand{\Perf}{{\it Perf}}
%%%\newcommand{\proof}{\flushleft{ \sc Proof. } \\ }
\newcommand{\Proof}[1]{\bigbreak\noindent{\bf Proof #1}\enspace}

%Q
%%%\newcommand{\qed}{{\hfill\vrule height6pt width6pt depth1pt\medskip}}
%\newcommand{\qed}{\sharp}
\newcommand{\QED}{\hspace{0.1in} \Box \vspace{0.1in}}

%R
\newcommand{\ra}{\rangle}
\newcommand{\ran}{{\rm ran}}
\newcommand{\rational}{\mathbb{Q}}
\newcommand{\real}{\mathbb{R}}

%S
\newcommand{\seq}{\subseteq}
%%%\newcommand{\square}{\hbox{\ \ \ \ \ \vrule\vbox{\hrule\phantom{o}\hrule}\vrule}}
% a small restriction:
\newcommand{\srestriction}{{\hbox{${\scriptstyle\,|\grave{}\,}$}}}
\newcommand{\supp}{\mathit{supp}}

%U
\newcommand{\up}{\uparrow}
\newcommand{\ucrz}{UCR_0}

%%%%%%%%%%%%%%%%%%%%%% Calligraphic font commands %%%%%%%%%%%%%%%%%%%%%%%%%%
\newcommand{\cA}{{\mathcal A}}
\newcommand{\cB}{{\mathcal B}}
\newcommand{\cC}{{\mathcal C}}
\newcommand{\cD}{{\mathcal D}}
\newcommand{\cE}{{\mathcal E}}
\newcommand{\cF}{{\mathcal F}}
\newcommand{\cG}{{\mathcal G}}
\newcommand{\cH}{{\mathcal H}}
\newcommand{\cI}{{\mathcal I}}
\newcommand{\cJ}{{\mathcal J}}
\newcommand{\cK}{{\mathcal K}}
\newcommand{\cL}{{\mathcal L}}
\newcommand{\cM}{{\mathcal M}}
\newcommand{\cN}{{\mathcal N}}
\newcommand{\cO}{{\mathcal O}}
\newcommand{\cP}{{\mathcal P}}
\newcommand{\cQ}{{\mathcal Q}}
\newcommand{\cR}{{\mathcal R}}
\newcommand{\cS}{{\mathcal S}}
\newcommand{\cT}{{\mathcal T}}
\newcommand{\cU}{{\mathcal U}}
\newcommand{\cV}{{\mathcal V}}
\newcommand{\cW}{{\mathcal W}}
\newcommand{\cX}{{\mathcal X}}
\newcommand{\cY}{{\mathcal Y}}
\newcommand{\cZ}{{\mathcal Z}}
%%%%%%%%%%%%%%%%%%%%%% inne %%%%%%%%%%%%%%%%%%%%%%%%%%%%%%%%%%%%%%%
\newcommand{\SqrFr}{\mathbb{SF}}
\newcommand{\Primes}{\mathit{Primes}}
\newcommand{\mathint}{\mathit{int}}
\newcommand{\mathcl}{\mathit{cl}}
\newcommand{\exampleGFnotK}{\mathit{Ex}}
%%%%%%%%%%%%%%%%%%%%%% Some Greek fonts %%%%%%%%%%%%%%%%%%%%%%%%%%%%%%%%%%%%%%%
\newcommand{\oo}{\omega}
\newcommand{\bb}{\beta}
\newcommand{\dd}{\delta}
\newcommand{\ee}{\varepsilon}
\newcommand{\kk}{\kappa}
%%%\newcommand{\th}{\theta}
%%%% makra specyficzne dla tej pracy:
\newcommand{\aideal}{\mathit{(a)}}
\newcommand{\aidealprime}{\mathit{(a^\prime)}}
\newcommand{\Afield}{\mathit{(A)}}
\newcommand{\topWithoutEmptyset}[1]{#1\setminus\lbrace\emptyset\rbrace}
\newcommand{\ND}{\mathsf{ND}}
\newcommand{\biMB}{bi-Marczewski-Burstin}
\newcommand{\tauEllentuck}{\tau_{\mathrm{EL}}}
\newcommand{\baseEllentuck}{\cB_{\mathrm{EL}}}
\newcommand{\ninomega}{n\in\omega}
\newcommand{\Hereditary}{\mathcal{H}}
%\renewcommand\abstractname{Summary}
%%%To print the date and time on each page
%%% comment out if not needed (next 14 lines)
\makeatletter
{\newcount\@hour}
{\newcount\@minute}
\def\timenow{\@hour=\time \divide\@hour by 60
\number\@hour:
  \multiply\@hour by 60 \@minute=\time
  \global\advance\@minute by -\@hour
  \ifnum\@minute<10 0\number\@minute\else
  \number\@minute\fi}
\def\ctimenow{\hfil{\tt \jobname.tex, \today~Time: \timenow }\hfil}
      \let\@oddfoot\ctimenow\let\@evenfoot\ctimenow
\makeatother

\pagestyle{myheadings}
\markboth{{\bf Andrzej Nowik}}{\bf \today}

% Types message, asks the user to type in a command, then
% defines \answer to be the input instead of executing it.
%%% comment out if not needed (next 13 lines)
%\typein[\answer]{Do you want to include comments? (y/n)}
%
%\newcommand{\annotation}[1]
%  {
%  \if\answer y {{\tt #1}}
%  \fi
%  }
%
%\if\answer y
%\typeout{I shall INCLUDE comments.}
%\else
%\typeout{Comments will be NOT shown.}
%\fi

% To see corrections comment next line and uncomment the second one
\newcommand{\correction}[2]{#1}
% \newcommand{\correction}[2]{#2}

\pagestyle{myheadings}
% To see corrections comment next line and uncomment the second one

\author{Andrzej Nowik}
\address[]{
Andrzej Nowik\\
University of Gda\'nsk \\
Institute of Mathematics \\
Wita Stwosza 57,
80--952 Gda\'nsk, Poland \\
}
\email[A.~Nowik]{andrzej.nowik@ug.edu.pl}

\begin{document}
\title[
% a brief title
On ideal $(a)$, revisited.
]{
%a full title
On ideal $(a)$, revisited.
}
\keywords{density topology, ideals}
\subjclass[2010]{03E05, 54D05, 54H05, 54A05, 11A41, 11B05, 11B25}

\begin{abstract}
\dummy
\end{abstract}

\maketitle

\section{Preliminaries}
\label{section:preliminaries}
The condition $(a)$ was first introduced by M. Grande 
%%%@@ pomijamy na razie lata - mozna je przeciez ,,odtworzyc'' z dat cytowanych publikacji
%%%in 2001 
in the paper
\cite{MarcinGrande}. Namely, he characterized the set of discontinuity points of some approximately continuous function f with property
$(s_1)$ (see \cite[Theorem 1]{MarcinGrande}) in terms of so called $(a)$ condition:
For each set $U\in\tau_d\setminus\lbrace\emptyset\rbrace$ contained in the closure $\cl(A)$ of the set $A$ the
set $U \cap A$ is nowhere dense in $U$.
In 
%%%2003 
\cite{GS} Z.Grande and E.Stro\'nska proved that the family of all sets satisfying
the condition $(a)$ is an ideal of sets included in both the ideal of
nowhere dense sets and in the $\sigma$-ideal of Lebesgue measure zero sets, 
which is $G_{\delta\sigma}$-generated, but not 
$F_{\sigma}$-generated. 
%%%@@ zbedne bo wyzej jest to sformulowane przypuszczalnie w bardziej zwarty sposob
%%%They also observed that every set satisfying condition $(a)$ is nowhere dense and of %%%Lebesgue measure zero. 
In \cite{FN2}
%%%2011 
M.Frankowska and A.Nowik proved that ideal $(a)$ is not $G_{\delta}$-generated
(it was a question in \cite{GS}).
Various further properties of the ideal $(a)$ were investigated in \cite{N}
%%%, see also [3].
where there was also defined a general version of the ideal for another
topologies. 
It turned out that the ideal $(a)$ is a special case of
a general definition: $\aideal = \aideal(\tau_e, \tau_d)$ 
, where $\tau_e$ is the standard Euclidean topology and 
$\tau_d$ is the standard density topology (see \cite{N} Theorem 2.4).
In said article it was proven that the ideal $\aideal(\tau_1, \tau_2)$
is equal to the intersection of all collections of nowhere dense sets
for all topologies between $\tau_1$ and $\tau_2$:
$\bigcap_{\tau_1 \subseteq \tau \subseteq \tau_2 \atop \tau \hbox{\tiny \it
topology}} \ND(\tau)$ (see Theorem 2.6, \cite{N}).
%%% TODO: czy tu nie powinno aby byc "The further(...)"
Further investigations of the properties of the ideal $(a)$ can be find in the paper  
\cite{FG}.
Notice also that we can consider the ideal $(a)$ defined on the Cantor
space $2^{\omega}$ instead the real line, and 
M.Frankowska and A.Nowik in \cite{FN1} proved that also in that case
the ideal $\aideal$ on the Cantor space is 
not $G_{\delta}$-generated (in fact, they proved implicite that the ideal $(a)$ is not
$F_{\sigma\delta}$ generated).

  We try to sketch the history of the next class of sets investigated
in this article. Namely, in \cite{GS}
the authors showed that for any closet set $E\subseteq \real$ we have
$E\setminus \Phi(E) \in (a)$, where $\Phi(E)$ denote a set of
density points of $E$. The collection of subsets of sets of this form was
denoted by $\aidealprime$ in \cite[Section 4]{FG}. The generalization of the notion $\aidealprime$ to any topologies was defined in \cite{N}. 
In \cite{FG} authors proved that the family $\aidealprime$
is not an ideal (\cite[Proposition 6]{FG}) and the sigma ideals generated by 
$\aidealprime$ and $\aideal$ are equal (\cite[Theorem 12]{FG}).

\section{Definitions and Notation}
\subsection{General notation}
If $A$ is any set then
$[A]^\omega$ denotes the
family of all infinite countable
subsets of the set $A$.
Quantifiers $\forall^\infty_{\ninomega}$ and $\exists^\infty_{\ninomega}$ denote 
$\exists_{m\in\omega} \forall_{n > m}$ and $\forall_{m\in\omega} \exists_{n > m}$,
respectively.
If $X$ is a Polish space then $\Perf(X)$ 
stands for the family of all perfect subsets of $X$.

\subsection{The Cantor space}
The Cantor space $2^\omega$ is the set of all infinite binary sequences with the metric 
%%%,which - to chyba moze byc zbedne
defined as follows: 
\[\rho((x_n), (y_n)) = \begin{cases}
     2^{-\min\lbrace i \in \omega\colon x_i \not= y_i\rbrace} & \text{if }(x_n) \not= (y_n)\\
     0      & \text{if }(x_n) = (y_n)
\end{cases}.\]
The base of the topology of the Cantor space $2^\omega$ consists of sets 
$\langle s \rangle = \lbrace x \in 2^\omega\colon x \restriction |s| = s\rbrace$ for any $s\in 2^{<\omega}$ (the set of all finite binary sequences) where $|s|$ denotes
the length of the binary sequence $s$.
\subsection{Ellentuck topology}
For explain some topological notions we refer for example \cite{En}.
If $s \in \finsub$, $U \in \infsub$ and $\max (s) < \min(U)$,
then by $[s, U]$ we denote $\{ A \in \infsub: s \seq A \seq s \cup U\}$.
We call such a set an {\it Ellentuck set}. We denote the collection of all
Ellentuck sets by $\baseEllentuck$ and this collection establish a base of a topology, called 
{\it Ellentuck topology}. We denote this topology 
by $\tauEllentuck$. This topology was well investigated.
It is well known that the collection of nowhere dense sets in the Ellentuck 
topology is equal to the  collection of first category sets in this topology.
%%%TODO: nie wiem czy mozna zastosowac taka ,,wyrzutnie'': "in the one" ??
%%% - a jednak zmienilem na "in this topology"
We call the nowhere dense sets in the Ellentuck topology 
{\it Ramsey null sets} and we have the following characterizations of
these sets: $X \subseteq \infsub$ is Ramsey null iff for every $[s, U]$ there exists $V \in [U]^\omega$
such that $X \cap [s, V] = \emptyset$ and the second characterization:
$X \subseteq \infsub$ is Ramsey null iff for every $[s, U]$ there exists $[t, V] \subseteq [s, U]$
such that $X \cap [s, V] = \emptyset$.
%%%  We say that a set $X \seq \infsub$ is Ramsey null
%%%(or shortly, X is $CR_0$) iff for every $[s, U]$, there exists $[s, V] \seq [s, U]$ such that
%%%$[s, V] \cap X = \emptyset$.  If $X \seq \ca$, then we will call it a Ramsey null
%%%set iff for every $[s, U]$, there is $[s, V] \seq [s, U]$,
%%%so that  $\{ \chi(x): x \in [s, V]\} \cap X = \emptyset$, where for $x \in \infsub$, $\chi(x)$ denotes its characteristic function.
\subsection{The density topology}
If $E \subseteq \real$ is a Lebesgue measurable set then by $\Phi(E)$ we denote the set of all denstiy
points of set $E$. It is well known that the collection of all Lebesgue measurable subsets of $\real$
such that $E \subseteq \Phi(E)$ is a topology called the \underline{density topology}, we denote
this topology by $\tau_d$.
\subsection{The category density topology}
Following \cite{PWBW} let us formulate a sequence
of definitions:
Assume that $\cF$ is a $\sigma$-field of subsets of $\real$
and $\cI \subseteq \cF$ is a proper $\sigma$-ideal of sets.
Suppose that $f$ is a $\cF$ measurable real function.
We say that a sequence $(f_n)_{n\in\omega}$
of $\cF$ measurable real functions \underline{converges
with respect to $\cI$ to the function $f$} iff
for every subsequence $(n_m)_{m\in\omega}$
there exists a subsequence $(n_{m_k})_{k\in\omega}$
such that 
$\real\setminus \lbrace x\in\real\colon \lim_{k\to\infty}
f_{n_{m_k}}(x) = f(x)\rbrace \in \cI$. We denote this fact
by $f_n \xrightarrow[n\to\infty]{\cI} f$.  
We say that $0$ is an \underline{$\cI$-density point}
of a set $A\in\cF$ iff 
$\chi_{(n \cdot A) \cap [-1, 1]} \xrightarrow[n\to\infty]{\cI} 0$. 
We say that $x_0$ is an $\cI$-density point of
$A\in\cF$ iff $0$ is an $\cI$-density point of
$A - x_0$.
If $A \in \cF$ then by $\Phi_{\cF, \cI} (A)$ we denote the set 
$\lbrace x \in \real\colon x \text{ is an } \cI \text{-density point of }A\rbrace$.
It is known (see \cite{PWBW}, Theorem 3) that the collection 
$\tau_{\cF, \cI} = \lbrace A \in \cF\colon A \subseteq \Phi_{\cF, \cI} (A)\rbrace$
is a topology on the real line. 
If we consider sets having the Baire property as the 
$\sigma$-field $\cF$, and for $\cI$ we take
the $\sigma$-ideal of meager sets, then
the topology $\tau_{\cF, \cI}$ is called
the \underline{category density topology}
and we denote this topology by $\cF_{\meager}$.

\subsection{Marczewski - Burstin representation}

\begin{definition}
For any nonempty family $\cF \subseteq P(X) \setminus \{\emptyset\}$
let us define
$S(\cF) = \{A\subseteq X\colon \forall_{P\in\cF} \exists_{Q\in \cF}
Q \subseteq P\cap A \vee Q \subseteq P\setminus A\}$
and 
$S^0(\cF) = \{A\subseteq X\colon \forall_{P\in\cF} \exists_{Q\in \cF}
Q \subseteq P\setminus A\}$. 
\end{definition}
It is well known that the collection $S^0(\cF)$ and $S(\cF)$
is a field and respectively, an ideal of subsets of $X$.
%%%@@TODO: nie mam pojecia czy to pojecie bedzie wykorzystywane dalej. W razie 
%%% czego mozna bedzie odkomentowac co nastepuje:
%We say that the family $\cF$ is an {\it inner}
%MB-representation (see \cite{BBRW},
%\cite{BBC1}, \cite{BBC2} or \cite{BBK})
%iff $\cF \subseteq S(\cF)$.

It is worth to note that the families $S(\Perf(X))$
and $S^0(\Perf(X))$ are the classical Marczewski sets
$s$ and $s_0$, respectively (introduced in 1935 in \cite{SM}).

%%%%%@@TODO: to tez ewentualnie mozna pozniej odkomentowac:
%A classical standard examples of MB reprezentations are:
%\begin{example}
%\label{example-measure-category}
%$S(\Perf^+) = \Lebesgue(\real)$, 
%$S^0(\Perf^+) = \negligible(\real)$ (Burstin, \cite{Bu}) and
%$S(\cM^+) = \Baire(\real)$,
%$S^0(\cM^+) = \meager(\real)$ (Brown and Elalaoui-Talibi, \cite{BET}),
%where $\Perf^+ = \Perf(\real)\setminus\negligible$ and
%$\cM^+$ is the family of subsets of the real line of the form $U\setminus F$, where
%$U$ is a nonempty open set and $F$ is an $F_{\sigma}$ meager set.
%\end{example}
%%%%%%%%%%%%%%%%%%%%%%%%%%%

If $\cF$ is a field of subsets of $X$ then define $\Hereditary(\cF) = \lbrace A \subseteq X \colon P(A) \subseteq \cF \rbrace$.

Notice that the field $S(\tau_e\setminus \lbrace \emptyset \rbrace)$
%%%defined via the Marczewski-Burstin representation:}\linebreak
%$\lbrace A \subseteq \real\colon 
%\forall_{U \in \tau_e \setminus \lbrace \emptyset \rbrace}
%\exists_{V \in \tau_e \setminus \lbrace \emptyset \rbrace \atop V \subseteq U}
%V \subseteq A \vee V \cap A = \emptyset\rbrace$
%%%%%%%%%%%%%%%%%%%%
is the field of sets of form $U \bigtriangleup E$
where $U$ is open and $E$ is nowhere dense
which is the same as the collection of sets $A$ such that
$\mathrm{Fr}(A)$ is nowhere dense. Notice also 
that if $\tau_1$ is a Hausdorff topology without isolated points
having a countable base, then 
\begin{equation}\label{equation:hereditary_s_tau}
\Hereditary(S(\tau_1\setminus \lbrace \emptyset \rbrace)) = \ND(\tau_1).
\end{equation}

The proof is rather folklore and uses an induction over elements
of the base $(V_n)_{n=0}^\infty$ of the topology $\tau_1$.

\section{Main part}

Let us recall the general version of the ideal $(a)$
(see \cite{N})
Suppose that $(X, \tau_1,\tau_2)$ is a bitopological space 
(i.e. $\tau_1$ and $\tau_2$ are topologies on $X$, 
see \cite{D}
%%%@@ albo... jakie jeszcze cytowanie tu by pasowalo?
)
such that $\tau_1$ is coarser than $\tau_2$:
$\tau_1 \subseteq \tau_2$. Then 
define $\aideal(\tau_1, \tau_2) = \{ A \subseteq X \colon
\forall_{U \in \topWithoutEmptyset{\tau_2}}
\exists_{W \in \tau_1} U \cap W \not= \emptyset \wedge
U \cap W \cap A = \emptyset\}$.

By \cite[Theorem 2.8]{N} we know that 
if $E$ is closed set in the topology $\tau_1$
then $E \setminus  \mathint_{\tau_2}(E) \in \aideal(\tau_1, \tau_2)$
Since for every closed set $E$ we have
$\mathint_{\tau_d}(E) = \Phi(E)$
we may define the following generalization of the family 
$\aidealprime$:
\begin{definition}
\label{definition:a_prime}
$\aidealprime(\tau_1, \tau_2) = \lbrace A \subseteq X\colon \exists_{E \subseteq X} 
X\setminus E \in \tau_1 \wedge A \subseteq E \setminus  \mathint_{\tau_2}(E) \rbrace$
\end{definition}

We have $\aidealprime(\tau_1, \tau_2) \subseteq \aideal(\tau_1, \tau_2)$.

Analogously to the definition of the ideal $\aideal(\tau_1, \tau_2)$ 
let us introduce the following notion:

\begin{definition}
\label{definition:field_A}
$\Afield(\tau_1, \tau_2) = \{ Z \subseteq X \colon
\forall_{U \in \tau_2 \setminus \{ \emptyset\}}
\exists_{W \in \tau_1} U \cap W \not= \emptyset \wedge
(U \cap W \cap Z = \emptyset \vee U \cap W \subseteq Z)\}$.
\end{definition}

Notice that $\Afield(\tau_1, \tau_2)$ is a field of sets.

Notice also that in the same way like (folklor) proof of 
the Equality (\ref{equation:hereditary_s_tau}) we can proof that 
\begin{fact}
\label{fact:hereditary_A}
Assume that $\tau_1$ is a Hausdorff topology without isolated points
having a countable base, then
$\Hereditary(\Afield(\tau_1, \tau_2)) = \aideal(\tau_1, \tau_2))$
\end{fact}

\begin{theorem}
\label{theorem:collapse-theorem}
%Denote by $\cK$ the following collection of subsets of the real line:
%$\lbrace A \subseteq \real \colon 
%\exists_{V\in\tau_e\setminus\lbrace\emptyset\rbrace}  
%\mathint (A \bigtriangleup V) = \emptyset\rbrace$.

Suppose that $\tau_1$ and $\tau_2$ are topologies on a set $X$ such
that $\tau_1 \subseteq \tau_2$.
The following conditions are equivalent:
\begin{enumerate}
\item $\tau_2 \cap \ND(\tau_1) = \lbrace \emptyset \rbrace$;
\item $\aideal(\tau_1, \tau_2) = \ND(\tau_1)$;
\item $\ND(\tau_1) \subseteq \ND(\tau_2)$.
\end{enumerate}
\end{theorem}
									
\begin{proof}  
%Notice that although the collection of sets with an empty interior is not an ideal,
%we have that $\cK = \lbrace (V \setminus B_1) \cup B_2\colon
%\mathint(B_1) = \mathint(B_2) = \emptyset, 
%V\in\topWithoutEmptyset{\tau_e}\rbrace$.
%%%TODO: czy jednak nie przydaloby sie tutaj jakies uzasadnienie?
$(1) \implies (2)$

It suffices to show the inclusion $\subseteq$, since
$\aideal(\tau_1, \tau_2) \subseteq \ND(\tau_1)$.
Suppose $A\in \ND(\tau_1)$ and choose
$W \in \tau_2\setminus\lbrace \emptyset \rbrace$.
By the assumption $(1)$ we have $W \not\in \ND(\tau_1)$, 
hence $\mathint_{\tau_1} \cl_{\tau_1} (W) \not= \emptyset$.
Since $A \in \ND(\tau_1)$ there exists 
$V\in \tau_1\setminus\lbrace\emptyset\rbrace$ such 
that $V \subseteq \mathint_{\tau_1} \cl_{\tau_1} (W) \setminus A$. 
Since $\emptyset \not= V \subseteq \cl_{\tau_1} (W)$, 
we have $W \cap V \not= \emptyset$. Moreover,
$W \cap V \cap A = \emptyset$. Therefore 
$A \in \aideal(\tau_1, \tau_2)$.

%Suppose $A \in \mathsf{ND}(\real)$ and let $W \in \tau_2\setminus\lbrace \emptyset \rbrace$.
%Then there exists $V\in\topWithoutEmptyset{\tau_e}$ such that
%$\mathint (W \bigtriangleup V) = \emptyset$. Since $A$ is nowhere dense, there
%exists $U \in \topWithoutEmptyset{\tau_e}$ such that
%$U \subseteq V$ and $U \cap A = \emptyset$. Then
%$\mathint(U \setminus W) \subseteq \mathint(V \setminus W) = \emptyset$,
%hence $U \cap W \not= \emptyset$. Thus $A \in \aideal(\tau_e, \tau_2)$.
%On the other hand, we always have 
%$\aideal(\tau_e, \tau_2) \subseteq \mathsf{ND}(\tau_e) = \mathsf{ND}(\real)$,
$(2) \implies (3)$.
It follows easily, since 
$\aideal(\tau_1, \tau_2) \subseteq \mathsf{ND}(\tau_2)$.

$\neg(1)\implies \neg(3)$.
Suppose that $W\in\tau_2\cap\mathsf{ND}(\tau_1) \setminus \lbrace\emptyset\rbrace$.
Obviously $W\not\in\ND(\tau_2)$ thus
$W \in \ND(\tau_1) \setminus \ND(\tau_2)$,
which finishes the proof.
\end{proof}

%Moreover, let us notice that the collection $\cK$ 
%is the collection of sets $A$ such that $\mathint(\cl(A)) \not=\emptyset$. 
%Indeed, we have that $\mathint(A \setminus \mathint(\cl(A))) = \emptyset$
%and 
%$\mathint(\mathint(\cl(A)) \setminus A) = \emptyset$, so 
%if $\mathint(\cl(A)) \not= \emptyset$ then $A\in \cK$.
%On the other hand, suppose that $A \in \cK$, so
%there exists $B_1, B_2 \subseteq \real$ and 
%$V\in\topWithoutEmptyset{\tau}$
%such that 
%$\mathint(B_1) = \mathint(B_2) = \emptyset$
%and  
%$A = (V \setminus B_1) \cup B_2$. 
%Then 
%$\mathint(\cl(A)) \supseteq 
%\mathint(\cl(V \setminus B_1)) = \mathint(\cl(V)) \supseteq V \not= \emptyset$.
%%%%%%%%%%%%%%%%%%%%%%%%%%%%%%%%%%%%%%%%%%%%
An immediate consequence of the previous theorem is the answer to the following
problem: at first, let us recall that in the definition of the standard ideal $\aideal$
we use the (measure) density topology. It is a natural question what kind of an ideal
we obtain if in the definition of the ideal $\aideal$ we would use the category density topology
instead of the measure one? Suprisingly it turned out that we get no new ideal, namely:
\begin{corollary}
\label{corollary:category-density-topology}
If $\cF_{\meager}$ is the category analogue of the density topology (see \cite{PWBW}) 
then 
$(a)(\tau_e, \cF_{\meager}) = \mathsf{ND}(\real)$.
\end{corollary}

\begin{proof}
This follows from the fact that 
$\cF_{\meager} \setminus\lbrace\emptyset\rbrace\subseteq \Baire\setminus\meager$
(see \cite{PWBW}, Remark 2) and
from Theorem \ref{theorem:collapse-theorem}
\end{proof}

\begin{corollary}
If we assume moreover that $\tau_1$ is a Hausdorff topology without isolated points
having a countable base then the conditions from Theorem \ref{theorem:collapse-theorem} are equivalent to
the condition: $\Afield(\tau_1, \tau_2) = \cS(\tau_1 \setminus \lbrace \emptyset \rbrace)$.
\end{corollary}

\begin{proof}
Assume that $\tau_2 \cap \mathsf{ND}(\tau_1) = \lbrace \emptyset \rbrace$.
It is evident that $\Afield(\tau_1, \tau_2) \subseteq \cS(\tau_1 \setminus \lbrace \emptyset \rbrace)$.
The argument that $\cS(\tau_1 \setminus \lbrace \emptyset \rbrace)\subseteq \Afield(\tau_1, \tau_2)$
''mimic'' the proof of implication $(1) \implies (2)$, but instead of the inclusion
$V \subseteq \mathint_{\tau_1} \cl_{\tau_1} (W) \setminus A$ we have the alternative
$V \subseteq \mathint_{\tau_1} \cl_{\tau_1} (W) \setminus A \vee V \subseteq \mathint_{\tau_1} \cl_{\tau_1} (W) \cap A$ 
and  we obtain $W \cap V \cap A = \emptyset \vee W \cap V \subseteq A$, so finally we
obtain $A \in \Afield(\tau_1, \tau_2)$.
On the other hand, assume that 
$\Afield(\tau_1, \tau_2) = \cS(\tau_1 \setminus \lbrace \emptyset \rbrace)$. 
Then by taking from both sides of this equation the operation $\Hereditary$
we obtain that 
$\Hereditary(\Afield(\tau_1, \tau_2)) = \Hereditary(\cS(\tau_1 \setminus \lbrace \emptyset \rbrace))$
therefore by Equality (\ref{equation:hereditary_s_tau}) 
and by Fact \ref{fact:hereditary_A} we obtain
$\aideal(\tau_1, \tau_2) = \mathsf{ND}(\tau_1)$.
\end{proof}

Let us formulate a ''dual'' theorem:

\begin{theorem}
Suppose that $\tau_1$ and $\tau_2$ are topologies on a set $X$ such
that $\tau_1 \subseteq \tau_2$.
Then we have implications $(1) \implies (2) \implies (3)$
where:
\begin{enumerate}
\item $\aideal(\tau_1, \tau_2) = \mathsf{ND}(\tau_2)$;
\item $\mathsf{ND}(\tau_2) \subseteq \mathsf{ND}(\tau_1)$.
\item $\tau_1 \cap \mathsf{ND}(\tau_2) = \lbrace \emptyset \rbrace$;
\end{enumerate}
\end{theorem}

\begin{proof}
$(1)\implies (2)$ is obvious, since
$\aideal(\tau_1, \tau_2) \subseteq \mathsf{ND}(\tau_1)$.

$\neg(3)\implies \neg(2)$.
Suppose that $W\in\tau_1\cap\mathsf{ND}(\tau_2) \setminus \lbrace\emptyset\rbrace$.
Obviously $W\not\in\mathsf{ND}(\tau_1)$ thus
$W \in \mathsf{ND}(\tau_2) \setminus \mathsf{ND}(\tau_1)$,
which finishes the proof.
\end{proof}

Notice that in the previous theorem we don't need to have
the implication $(3) \implies (2)$. The counterexample 
is the density topology ideal $\aideal$, for $\tau_1 = \tau_e$
and $\tau_2 = \tau_d$. Indeed, we have 
that $\mathsf{ND}(\tau_2)$ is a collection of Lebesgue null sets, 
so $\tau_1 \cap \mathsf{ND}(\tau_2) = \lbrace \emptyset \rbrace$,
but of course neither $\mathsf{ND}(\tau_2) \subseteq \mathsf{ND}(\tau_1)$ 
nor $\aideal(\tau_1, \tau_2) = \mathsf{ND}(\tau_2)$.

%%%@@TODO: tutaj twierdzenie o przekroju topologii, niewykluczone 
%%% ze powinno sie je przeniesc gdzie indziej, aha i dopisac ,,proze'':

\begin{theorem}
In said article it was proven that the ideal $\aideal(\tau_1, \tau_2)$
is equal to the intersection of all collections of nowhere dense sets
for all topologies between $\tau_1$ and $\tau_2$:
There are no topologies $\lbrace \tau_i \colon i = 1,\ldots, n\rbrace$
such that $\forall_{i = 1,\ldots, n} \tau_e \subseteq \tau_i \subseteq \tau_d$
and 
$\aideal = \bigcap_{i=1,\ldots,n} \ND(\tau_i)$.
\end{theorem}

\begin{proof}
By way on contradiction suppose that 
$\aideal = \bigcap_{i=1,\ldots,n} \ND(\tau_i)$ for some topologies 
 $\tau_i$ such that $\tau_e \subseteq \tau_i \subseteq \tau_d$.
Consider the partially ordered set 
$\mathbb{P} = \langle \tau_e\setminus\lbrace\emptyset\rbrace, \subseteq\rangle$
and define:
$\mathfrak{D}_i = \lbrace W \in \tau_d\setminus\lbrace\emptyset\rbrace \colon
\mathit{int}_{\tau_i}(W) = \emptyset \vee \forall_{V \in \tau_d\setminus\lbrace\emptyset\rbrace \atop V \subseteq W} \mathit{int}_{\tau_i} V \not= \emptyset\rbrace$.
It is easy to see that $\mathfrak{D}_i$ is a dense subset of 
the set $\mathbb{P}$, i.e. for each $W \in \mathbb{P}$ we can find
$V \in \mathbb{P}$, $V \subseteq W$ and such that $V \in \mathfrak{D}_i$
(...)
\end{proof}

It is easy to see that in the definition of $\aideal(\tau_1, \tau_2)$
(see Section \ref{section:preliminaries})
we may assume that instead of taking $W \in \tau_1$
we can take $W$ from any fixed base $\cB$ of the topology
$\tau_1$. However, we cannot assume that (analogously)
the set $U$ can be taken only from (any fixed) base of
the topology $\tau_2$. The counterexample is based 
on Example \ref{example:ellentuck}

%%%% rozbudowany przyklad:
\begin{example}
There exists a Borel (in fact,
a $G_{\delta}$) %%%@@TODO: upewnic sie ze ta klasa borelowska faktycznie to gdelta!
set $Z \subseteq \infsub$ such that 
$Z \in \aideal(\tau_e, \baseEllentuck)$, but
$Z \not\in \aideal(\tau_e, \tauEllentuck)$.
\end{example}

\begin{proof}
Let $(A_n)_{n\in\omega}$ be a fixed partition of $\omega$ into infinite sets. 
Let us define a set $Z$:
$Z = \lbrace A\in\infsub \colon \forall_{n\in\omega} |A_n \cap A| = 1\rbrace$.
Let us check that $Z\in\aideal(\tau_e, \baseEllentuck)$.
Suppose that $[s, A]$ is an Ellentuck set, i.e. $s\in\finsub$, $A\in\infsub$, $\max s < min A$. 
If $[s, A] \cap Z = \emptyset$ then
the proof is complete. On the other hand suppose that there exists $B \in [s, A] \cap Z$. Then $s \subseteq B \in Z$, so $\forall_{
\ninomega} |A_n \cap s| \leq 1$.
Suppose that $\exists_{n_0\in\omega} |A_{n_0} \cap A| \geq 2$. Let us choose $k_0 < k_1$, $k_0, k_1 \in A_{n_0} \cap A$, observe that $\max s < k_0 < k_1$ and define:
$t \in 2^{<\omega}$, 
$\dom(t) = \lbrace 0, \ldots, k_1\rbrace$, by  
\[t(i) = \begin{cases}
     1 & \text{ if } i \in s \cup \lbrace k_0, k_1 \rbrace \\
     0 & \text{ in the other case }
\end{cases}.\]
We have $\langle t \rangle \cap [s, A] \not= \emptyset$. Indeed, let us define $D = s \cup \lbrace k_0, k_1 \rbrace \cup (A \setminus \lbrace 0, \ldots, k_1\rbrace) = 
s \cup \lbrace k_0 \rbrace \cup (A \setminus \lbrace 0, \ldots, k_1 - 1\rbrace)$.
Then $D \in \infsub$, $D \in [s, A]$ (since $s \subseteq D \subseteq s \cup A$) and $D \in \langle t \rangle$. Since $t(k_0) = t(k_1) = 1$ we have $Z \cap \langle t \rangle = \emptyset$.
On the other hand, suppose that $\forall_{\ninomega} |A \cap A_n| \leq 1$. Then $\exists^\infty_n A \cap A_n \not= \emptyset$.
Let us define 
$n_0 = \min\lbrace \ninomega \colon A \cap A_n \not= \emptyset \wedge A_n \cap s = \emptyset \rbrace$.
Let $\lbrace k_0 \rbrace = A \cap A_{n_0}$. Observe that $\max s < k_0$. Let us define $t \in 2^{<\omega}$ by $\dom(t) = \lbrace 0, \ldots, k_0\rbrace$, 
$t(i) = \begin{cases}
     1 & \text{ if } i \in s \cup A \setminus \lbrace k_0 \rbrace \\
     0 & \text{ in the other case }
\end{cases}$.
Define $D = s \cup (A \setminus \lbrace k_0 \rbrace)$. Then $D \in \infsub$, $D \in [s, A]$ (since $s \subseteq D \subseteq s \cup A$) and since $D \cap \lbrace 0, \ldots, k_0 \rbrace = s \cup (A \cap \lbrace 0, \ldots, k_0 - 1 \rbrace$ we have $D \in \langle t \rangle$. So $\langle t \rangle \cap [s, A] \not= \emptyset$. Suppose that $Z \cap \langle t \rangle \cap [s, A] \not= \emptyset$ and choose $B \in Z \cap \langle t \rangle \cap [s, A]$. Since $B \in Z$ we have $|B \cap A_{n_0}| = 1$. From the definiton of $n_0$ we know that $A_{n_0} \cap s = \emptyset$. Since $B \in [s, A]$ we have $B \subseteq s \cup A$. Hence $B \cap A_{n_0} \subseteq (s \cap A_{n_0}) \cup (A_{n_0} \cap A) = A_{n_0} \cap A = \lbrace k_0 \rbrace$. Therefore $B \cap A_{n_0} = \lbrace k_0 \rbrace$. But $t(k_0) = 0$, which is a contradiction with $B \in \langle t \rangle$.
This proves that 
$Z \in \aideal(\tau_e, \baseEllentuck)$.

Let us check now that 
$Z \not\in \aideal(\tau_e, \tauEllentuck)$.
Let us define:
\[
W = \bigcup_{B \in Z} [B]^\omega = \lbrace B\in\infsub\colon \forall_{\ninomega} 
|A_n \cap B| \leq 1\rbrace.
\]
Suppose that $t\in 2^{<\omega}$ is such that
$\langle t \rangle \cap W \not= \emptyset$.
Let $k_0$ be such that $\dom(t) = \lbrace 0, \ldots, k_0\rbrace$ and 
define: $a = \lbrace i \in \dom(t) \colon t(i) = 0 \rbrace$ 
and 
$b = \lbrace i \in \dom(t) \colon t(i) = 1 \rbrace$. Since 
$\langle t \rangle \cap W \not= \emptyset$ let us choose
any $E \in \langle t \rangle \cap W$. Thus 
$\forall_{\ninomega} |E \cap A_n| \leq 1$. But $b \subseteq E$,
hence $\forall_{\ninomega} |b \cap A_n| \leq 1$.
Let us denote $F = \lbrace \ninomega\colon b\cap A_n \not= \emptyset\rbrace$.
For each $n\in \omega$ choose $m_n\in\omega$ by the following way:
\[m_n = \begin{cases}
  \text{the unique element of } b \cap A_n   & \text{ if }
  n\in E\\
  \min A_n \setminus \lbrace 0, \ldots, k_0\rbrace & \text{ if }
  n\not\in E \\
\end{cases}.\]
Define $D = \lbrace m_n \colon n \in \omega\rbrace$.
Then $D \in Z$, hence $D \in W$.
We have $b\subseteq D$. Indeed, if $k\in b$ then
there exists $n(k)$ such that $k \in A_{n(k)}$.
Therefore $b \cap A_{n(k)} \not= \emptyset$, hence
$\lbrace m_{n(k)} \rbrace = b \cap A_{n(k)}$.
But $m_{n(k)} \in D$ and moreover $k = m_{n(k)}$
(since $k\in b \cap A_{n(k)}$), therefore
$k\in D$.
Let us check that $a \cap D = \emptyset$.
By way of contradiction, suppose that $m_n \in a$
for some $n\in\natural$. Since $a\cap b = \emptyset$
and for $n\in E$ we have that $m_n \in b$
hence $n \not\in E$. We know that 
$m_n \in A_n \setminus \lbrace 0, \ldots, k_0\rbrace$.
But $a \subseteq \lbrace 0, \ldots, k_0\rbrace$,
which is a contradiction.
Finally $D\in \langle t \rangle$, what
proves that $W \cap \langle t \rangle \cap Z \not= \emptyset$.
So indeed $A \not\in \aideal (\tau_e, \tauEllentuck)$.
\end{proof}

%%%% koniec rozbudowanego przykladu

\subsection{Examples}
\begin{example}
$X = \real$,
$\tau_1 = \tau_e$ and
$\tau_2 = \cF_{\meager}$
where $\cF_{\meager}$ is the category analogue of the density topology (see \cite{PWBW}).
We argued (see Corollary \ref{corollary:category-density-topology}) that we have 
$(a)(\tau_e, \cF_{\meager}) = \mathsf{ND}(\real)$.
\end{example}
\begin{example}
\label{example:ellentuck}
$X = [\omega]^\omega$,
$\tau_1$ is a standard topology (on the Baire space of infinite natural sequences) and
$\tau_2 = \tauEllentuck$, where $\tauEllentuck$ is the Ellentuck topology.
\end{example}

\subsection{$\aideal$ via Marczewski - Burstin schema}

We may consider the definition of the ideals $\aideal$
as a special kind of general schema via Marczewski - Burstin.
Analogously to the notion of bitopological spaces we call define
a {\it\biMB{}} system:

\begin{definition}
Suppose that $\cF_1$ and $\cF_2$ are collections of nonempty
subset of a set $X$. 
%%%@@TODO: a moze lepiej by brzmialo: analogously like...?
Analogously to the notion of bitopological spaces we call a structure
$\langle \cF_1, \cF_2 \rangle$ a {\it\biMB{}} system.
Notice that we don't require inclusion $\cF_1 \subseteq \cF_2$.
To avoid trivial cases assume that $\bigcup \cF_1 = X$.
\end{definition}

\begin{definition}
If $\langle \cF_1, \cF_2 \rangle$ is a {\it\biMB{}} system,
then define 
$\aideal(\cF_1, \cF_2) = \lbrace A \subseteq \real\colon 
\forall_{W\in \cF_2} \exists_{U \in \cF_1}
W \cap U \not= \emptyset \wedge W \cap U \cap A = \emptyset
\rbrace$ 
\end{definition}

However, such defined collection $\aideal(\cF_1, \cF_2)$
in general case need not be ever an ideal.
%%%@@TODO: podac tu jakis ciekawy - naturalny - kontrprzyklad.

Nevertheless, let us notice that the collection
$\aideal(\tau_e\setminus\lbrace\emptyset\rbrace, \Perf)$ is an ideal. Namely we have:

\begin{theorem}
\label{theorem:a_tau_e_perf_i}
Suppose that $\cI \subseteq P(\real)$ is a family of sets such that 
if $C\subseteq \real$ is a countable set and $F\in\cI$ then $F \cup C \in \cI$. 
Then $\aideal(\tau_e\setminus\lbrace\emptyset\rbrace, \Perf\setminus\cI) = \lbrace A \subseteq \real \colon \cl(A) \in \cI \rbrace$.
\end{theorem}

\begin{proof}
$\Rightarrow$ Suppose that $A\subseteq \real$ is such that $\cl(A) \not\in\cI$. Then $\cl(A) = P \cup C$, where $P$ is a perfect set and $C$ is a countable set. By the assumption on the family $\cI$ we concluse that $P\not\in\cI$. So we have $\forall_{W \in \tau_e} W \cap P \not= \emptyset \implies W \cap P \cap A \not= \emptyset$, therefore $A \not\in \aideal(\tau_e\setminus\lbrace\emptyset\rbrace, \Perf\setminus\cI)$.

$\Leftarrow$ Assume that $\cl(A)\in\cI$. Then $P \setminus \cl(A) \not=\emptyset$, so if we define $W = \real\setminus\cl(A)$ then we have $W \cap P \not= \emptyset$ and $W \cap P \cap A = \emptyset$, therefore  
$A \in \aideal(\tau_e\setminus\lbrace\emptyset\rbrace, \Perf\setminus\cI)$.
\end{proof}

\begin{corollary}
$\aideal(\tau_e\setminus\lbrace\emptyset\rbrace, \Perf)$ is an ideal.
\end{corollary}
\begin{proof}
It suffices to put $\cI =$ countable subsets of $\real$ in Theorem \ref{theorem:a_tau_e_perf_i}.
\end{proof}

Let us formulate a general theorem:

\begin{theorem}
Suppose that $\langle \cF_1, \cF_2 \rangle$ is a \biMB{} system.
Assume that 
\begin{enumerate}	
\item
\label{mb-condition-subset}
  $\cF_1 \subseteq \cF_2$
\item 
\label{mb-condition-f1}
  $\forall_{E,F \in \cF_1} E \cap F \not= \emptyset \implies E \cap F \in \cF_1$
%%%@@TODO: tu chyba wystarczy (ale trzeba to przerachowac)
% aby byl spelniony slabszy warunek, mianowicie:
% \forall_{E,F \in \cF_1} E \cap F \not= \emptyset \implies E \cap F 
% = \bigcup \cG$, where $\emptyset \not= \cG \subseteq \cF_1$.  
\item
\label{mb-condition-f1f2}
  $\forall_{E \in cF_1} \forall_{W \in \cF_2} E \cap W \not= \emptyset \implies E \cap W \in \cF_2$  
\end{enumerate}	
Then the collection
$\aideal(\cF_1, \cF_2) = \lbrace A \subseteq \real\colon 
\forall_{W\in \cF_2} \exists_{U \in \cF_1}
W \cap U \not= \emptyset \wedge W \cap U \cap A = \emptyset
\rbrace$ 
is an ideal. 
\end{theorem}
\begin{proof}
It suffices only to check that if $A, B \in \aideal(\cF_1, \cF_2)$ then
$A\cup B \in \aideal(\cF_1, \cF_2)$.
So suppose that $W \in \cF_2$, then there exists $U\in\cF_1$
such that 
$W \cap U \not= \emptyset$ and $W \cap U \cap A = \emptyset$.
By condition \ref{mb-condition-f1f2} 
we know that $W \cap U \in \cF_2$ so
we conclude that
there exists a set $V \in \cF_1$ such that
$W \cap U \cap V \not= \emptyset$ and $W \cap U \cap V \cap A = \emptyset$.
Since by condition \ref{mb-condition-f1} we know that 
$U \cap V \in \cF_1$, this ends the proof.
\end{proof}

\begin{corollary}
Both collections: $\aideal(\tau_e, \baseEllentuck)$ and
$\aideal(\tau_e, \tauEllentuck)$ are ideals.
\end{corollary} 

%%%%%%%%%%%%%%%%%%%%%%%%%%%%%%%%%%%%%%%%%%%%%%%%%%%%%%%%%%%%%%%%%%%%%%%%%%%%%
%                        The Bibliography                                   %
%%%%%%%%%%%%%%%%%%%%%%%%%%%%%%%%%%%%%%%%%%%%%%%%%%%%%%%%%%%%%%%%%%%%%%%%%%%%%
\begin{thebibliography}{10}
\bibitem[BT]{BT}
  M.K. Bose, R. Tiwari, \emph{On increasing sequences of topologies
on a set}, Riv. Mat. Univ. Parma, \textbf{7} (7), (2007), 173--183.

\bibitem[D]{D}
  B. P. Dvalishvili, \emph{Bitopological Spaces: Theory, 
Relations with Generalized Algebraic Structures, and Applications}.
North-Holland Mathematics Studies, \textbf{199}. 
Elsevier Science B.V., Amsterdam, 2005.

\bibitem[MG]{MarcinGrande} M.~Grande, \emph{On the sets of discontinuity points of functions satisfying some approximate
quasi- continuity conditions}, Real Anal. Exchange \textbf{27(2)} (2001-2002), 773--781.

\bibitem[En]{En}
R.Engelking \emph{General Topology}, Heldermann Verlag, Berlin, 1989.

\bibitem[FG]{FG}
M. Frankowska, S.G\l{}\c{a}b, 
\emph{On some $\sigma$-ideal without ccc}, Colloquium Mathematicum \textbf{158 (1)} (2019), 127--140.

\bibitem[FN1]{FN1} M. Frankowska, A. Nowik, \emph{On some ideal defined by density topology in the Cantor set},
Georgian Math. J. \textbf{19, No. 1}, 93-99 (2012).

\bibitem[FN2]{FN2} M. Frankowska, A. Nowik, \emph{The ideal (a) is not $G_\delta$ generated}, 
Coll. Math. vol. \textbf{125, no.1} (2011), 7-14.

\bibitem[G]{G}
  M.Grande {\em On the measurability of functions satisfying
some approximate quasicontinuity conditions},
{Real Analysis Exchange} {30 (2004/05)}, no. 1, 1 -- 10.

\bibitem[Gl]{Gl}
S.G\l{}\c{a}b
{\em Descriptive properties related to porosity and density for compact sets on the real line,}
Acta Math. Hungar., 116 (1-2) (2007), 61--71.

\bibitem[GS]{GS}
  Z.Grande, E.Stro\'nska, \emph{On an ideal of linear sets},
Demonstratio Mathematica, Vol. 36, No 2, (2003), 
307 -- 311.

\bibitem[Kelly]{Kelly}
J.C.Kelly, \emph{Bitopological spaces.} Proc. Lond. Math. Soc. 
\textbf{(3) 13} (1963), 71--89.

\bibitem[Ke]{Ke}
{A. S. Kechris}, {\em Classical Descriptive Set Theory}, Springer, Berlin,
1995. 

\bibitem[N]{N}
  A. Nowik, \emph{Notes on the ideal $(a)$},
{Tatra Mountains Mathematical Publications,} 46 (2010), 41--45.

\bibitem[Pl]{Pl} S.Plewik, \emph{On completely Ramsey sets}, 
Fund. Math. \textbf{127} (1986), 127-132.

\bibitem[PWBW]{PWBW}
W. Poreda, E. Wagner-Bojakowska, W\l{}adys\l{}aw Wilczy\'{n}ski
{\em A category analogue of the density topology}
Fundamenta Mathematicae 125 (1985), 167--173.

\bibitem[SM]{SM}{E.Szpilrajn-Marczewski}, \emph{Sur un classe de
fonctions de M. Sierpi\'nski et la class correspondante d\'{}ensembles},
Fund. Math. 24 (1935), 17--34.

\bibitem[T]{T}
 F.D.Tall, {\em The density topology,}
{Pacific J. Math.,} Vol 62, Number 1 (1976), 275--284
\end{thebibliography}

\begin{center}
\ctimenow
\end{center}

\end{document}

%%%%@@ to jest zbedne z uwagi na twierdzenie \ref{theorem:a_tau_e_perf_i}:
Let $A, B \in \aideal(\tau_e\setminus\lbrace\emptyset\rbrace, \Perf)$,
and let $P$ be a perfect set. 
There exist $W_A \in \tau_e$ such that 
$W_A \cap P \not= \emptyset \wedge W_A \cap P \cap A = \emptyset$.
Let $P_1 = \cl_{\tau_e}(P \cap W_1)$. It is well know that such defined
$P_1$ is also a perfect set. So let $W_B \in \tau_e$ 
be such that 
$W_B \cap P_1 \not= \emptyset \wedge W_B \cap P_1 \cap B = \emptyset$.
Define $W = W_A \cap W_B$. Then we have:
$W \cap P = W_B \cap W_A \cap P$, 
but since $W_B \cap \cl_{\tau_e}(P \cap W_A)$
we have $W_B \cap P \cap W_A \not= \emptyset$,
hence $W \cap P \not= \emptyset$.
Moreover, we have
$W \cap P \cap (A \cup B) = (W \cap P \cap A) \cup (W \cap P \cap B) \subseteq$
$(W_A \cap P \cap A) \cup (W_B \cap W_A \cap P \cap B) =$
$W_B \cap W_A \cap P \cap B \subseteq W_B \cap \cl_{\tau_e} (P \cap W_A) \cap B =$
$W_B \cap P_1 \cap B = \emptyset$.
\end{proof}
%%%%%%%%%%%%%%%%%%%%%%%%%%%%%%%%%%%