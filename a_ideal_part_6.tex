%%%%%%%%%%%%%%%%%%%%%%%%%%%%%%%%%%%%%%%%%%%%%%%%%%%%%%%%%%%%%%%%%%%%%%%%%
%                                                                       %
%                  praca o ideale (a)                                   %
%                                                                       %
%%%%%%%%%%%%%%%%%%%%%%%%%%%%%%%%%%%%%%%%%%%%%%%%%%%%%%%%%%%%%%%%%%%%%%%%%
%\documentclass[12pt]{article}
%%% aby uzyc amsart w pelnej krasie:
% 1. odkomentowac \address i \email
% 2. odkomentowac \keywords{} \subjclass
\documentclass[12pt]{amsart}
\usepackage{amssymb}
\usepackage{amsmath}
\usepackage{latexsym}
\usepackage{amsfonts}
\usepackage{enumerate}
%\usepackage[mathscr]{eucal}
\usepackage[mathscr]{euscript}
\usepackage{eqlist}
\usepackage{amsthm}
%%%%%%%%%%%%%%% Polish letter packages %%%%%%%%%%%%%%%%%%%%%%%%%%%%%%%%
\usepackage[polish]{babel}
\usepackage[utf8]{inputenc}
\usepackage{t1enc}
%%% pakiet do generowania belkotu (wypelniacza):
\usepackage{lipsum}
%---------------------------------------------------------------------
% Theorems 
%---------------------------------------------------------------------
 
% Theorem style 'plain' are for: Theorem, Lemma, Corollary,
% Proposition, Conjecture, Criterion, Algorithm
\theoremstyle{plain}
\newtheorem{theorem}{Theorem}[section]
\newtheorem{lemma}[theorem]{Lemma}
\newtheorem{corollary}[theorem]{Corollary}
\newtheorem{conclusion}[theorem]{Corollary}
\newtheorem{claim}[theorem]{Claim}
\newtheorem{fact}[theorem]{Fact}
\newtheorem{proposition}[theorem]{Proposition}
\newtheorem{axiom}{Axiom}
% Theorem style 'definition' are for: Definition, Condition, Problem,
% Example
\theoremstyle{definition}
\newtheorem{definition}[theorem]{Definition}
\newtheorem{example}[theorem]{Example}
\newtheorem{exercise}{Exercise}
\newtheorem*{solution}{Solution}
\newtheorem{remark}[theorem]{Remark}
\newtheorem{Problem}[theorem]{Problem}
% Theorem style 'remark' are for: Remark, Note, Notation, Claim,
% Summary, Acknowledgement, Case, Conclusion
\theoremstyle{remark}
\newtheorem*{notation}{Notation}
\newtheorem*{acknowledgement}{Acknowledgement}

%%%%%% makra:

%A
\newcommand{\afc}{AFC}
\newcommand{\afcbar}{\overline{AFC}}
\newcommand{\arr}{\rightarrow}
\newcommand{\Arr}{\Rightarrow}

%B
\newcommand{\baire}{\omega^{\omega}}
\newcommand{\Baire}{\mathfrak{Baire}}
\newcommand{\Bor}{\mbox{${\mathcal B}or$}}
%Previous seems to be much finer than next.
%\newcommand{\Bor}{{\it Bor}}
\newcommand{\borelucrz}{Borel-UCR_0}

%C
\newcommand{\ca}{2^{\omega}}
\newcommand{\cantor}{\ca}
\newcommand{\Card}[1]{\Vert #1 \Vert}
\newcommand{\cl}{\mathit{cl}}

%D
\newcommand{\dom}{{\rm dom}}
\newcommand{\dummy}{{\tt Blah blah blah}}

%E
\newcommand{\Even}{\hbox{\rm \tiny Even}}

%F
\newcommand{\finsub}{[\omega]^{<\omega}}
\newcommand{\forces}{\mathrel{\|}\joinrel\mathrel{-}}

%G
\newcommand{\Graph}{\hbox{\it Graph}}

%H
\newcommand{\homeomorphic}{\approx}

%I
\newcommand{\incr}{\omega^{\uparrow \omega }}
\newcommand{\infsub}{[\omega]^{\omega}}

%L
\newcommand{\la}{\langle}

%M
\newcommand{\meager}{{\mathcal{MGR}}}
\newcommand{\minideal}{${\cal F}_{\hbox{\rm \scriptsize min}}(\neg
  D)\;$}

%N
\newcommand{\neglig}{{\cal N}}
\newcommand{\nnatural}{\mathbb{N}}

%O
\newcommand{\Odd}{\hbox{\rm \tiny Odd}}

%P
\newcommand{\Part}{{\it Part}}
\newcommand{\Perf}{{\it Perf}}
%%%\newcommand{\proof}{\flushleft{ \sc Proof. } \\ }
\newcommand{\Proof}[1]{\bigbreak\noindent{\bf Proof #1}\enspace}

%Q
%%%\newcommand{\qed}{{\hfill\vrule height6pt width6pt depth1pt\medskip}}
%\newcommand{\qed}{\sharp}
\newcommand{\QED}{\hspace{0.1in} \Box \vspace{0.1in}}

%R
\newcommand{\ra}{\rangle}
\newcommand{\ran}{{\rm ran}}
\newcommand{\rational}{\mathbb{Q}}
\newcommand{\real}{\mathbb{R}}

%S
\newcommand{\seq}{\subseteq}
%%%\newcommand{\square}{\hbox{\ \ \ \ \ \vrule\vbox{\hrule\phantom{o}\hrule}\vrule}}
% a small restriction:
\newcommand{\srestriction}{{\hbox{${\scriptstyle\,|\grave{}\,}$}}}
\newcommand{\supp}{\mathit{supp}}

%U
\newcommand{\up}{\uparrow}
\newcommand{\ucrz}{UCR_0}

%%%%%%%%%%%%%%%%%%%%%% Calligraphic font commands %%%%%%%%%%%%%%%%%%%%%%%%%%
\newcommand{\cA}{{\mathcal A}}
\newcommand{\cB}{{\mathcal B}}
\newcommand{\cC}{{\mathcal C}}
\newcommand{\cD}{{\mathcal D}}
\newcommand{\cE}{{\mathcal E}}
\newcommand{\cF}{{\mathcal F}}
\newcommand{\cG}{{\mathcal G}}
\newcommand{\cH}{{\mathcal H}}
\newcommand{\cI}{{\mathcal I}}
\newcommand{\cJ}{{\mathcal J}}
\newcommand{\cK}{{\mathcal K}}
\newcommand{\cL}{{\mathcal L}}
\newcommand{\cM}{{\mathcal M}}
\newcommand{\cN}{{\mathcal N}}
\newcommand{\cO}{{\mathcal O}}
\newcommand{\cP}{{\mathcal P}}
\newcommand{\cQ}{{\mathcal Q}}
\newcommand{\cR}{{\mathcal R}}
\newcommand{\cS}{{\mathcal S}}
\newcommand{\cT}{{\mathcal T}}
\newcommand{\cU}{{\mathcal U}}
\newcommand{\cV}{{\mathcal V}}
\newcommand{\cW}{{\mathcal W}}
\newcommand{\cX}{{\mathcal X}}
\newcommand{\cY}{{\mathcal Y}}
\newcommand{\cZ}{{\mathcal Z}}
%%%%%%%%%%%%%%%%%%%%%% inne %%%%%%%%%%%%%%%%%%%%%%%%%%%%%%%%%%%%%%%
\newcommand{\SqrFr}{\mathbb{SF}}
\newcommand{\Primes}{\mathit{Primes}}
\newcommand{\mathint}{\mathit{int}}
\newcommand{\mathcl}{\mathit{cl}}
\newcommand{\exampleGFnotK}{\mathit{Ex}}
%%%%%%%%%%%%%%%%%%%%%% Some Greek fonts %%%%%%%%%%%%%%%%%%%%%%%%%%%%%%%%%%%%%%%
\newcommand{\oo}{\omega}
\newcommand{\bb}{\beta}
\newcommand{\dd}{\delta}
\newcommand{\ee}{\varepsilon}
\newcommand{\kk}{\kappa}
%%%\newcommand{\th}{\theta}
%%%% makra specyficzne dla tej pracy:
\newcommand{\aideal}{\mathit{(a)}}
\newcommand{\aidealprime}{\mathit{(a^\prime)}}
\newcommand{\Afield}{\mathit{(A)}}
\newcommand{\topWithoutEmptyset}[1]{#1\setminus\lbrace\emptyset\rbrace}
\newcommand{\ND}{\mathsf{ND}}
\newcommand{\biMB}{bi-Marczewski-Burstin}
\newcommand{\tauEllentuck}{\tau_{\mathrm{EL}}}
%\renewcommand\abstractname{Summary}
%%%To print the date and time on each page
%%% comment out if not needed (next 14 lines)
\makeatletter
{\newcount\@hour}
{\newcount\@minute}
\def\timenow{\@hour=\time \divide\@hour by 60
\number\@hour:
  \multiply\@hour by 60 \@minute=\time
  \global\advance\@minute by -\@hour
  \ifnum\@minute<10 0\number\@minute\else
  \number\@minute\fi}
\def\ctimenow{\hfil{\tt \jobname.tex, \today~Time: \timenow }\hfil}
      \let\@oddfoot\ctimenow\let\@evenfoot\ctimenow
\makeatother

\pagestyle{myheadings}
\markboth{{\bf Andrzej Nowik}}{\bf \today}

% Types message, asks the user to type in a command, then
% defines \answer to be the input instead of executing it.
%%% comment out if not needed (next 13 lines)
%\typein[\answer]{Do you want to include comments? (y/n)}
%
%\newcommand{\annotation}[1]
%  {
%  \if\answer y {{\tt #1}}
%  \fi
%  }
%
%\if\answer y
%\typeout{I shall INCLUDE comments.}
%\else
%\typeout{Comments will be NOT shown.}
%\fi

% To see corrections comment next line and uncomment the second one
\newcommand{\correction}[2]{#1}
% \newcommand{\correction}[2]{#2}

\pagestyle{myheadings}
% To see corrections comment next line and uncomment the second one

\author{Andrzej Nowik}
\address[]{
Andrzej Nowik\\
University of Gda\'nsk \\
Institute of Mathematics \\
Wita Stwosza 57,
80--952 Gda\'nsk, Poland \\
}
\email[A.~Nowik]{andrzej.nowik@ug.edu.pl}

\begin{document}
\title[
% a brief title
On ideal $(a)$, revisited.
]{
%a full title
On ideal $(a)$, revisited.
}
\keywords{density topology, ideals}
\subjclass[2010]{03E05, 54D05, 54H05, 54A05, 11A41, 11B05, 11B25}

\begin{abstract}
\dummy
\end{abstract}

\maketitle

\section{Preliminaries}
\label{section:preliminaries}
The definition of condition $(a)$ was first introduced by Marcin Grande in 2001 in the paper
\cite{MarcinGrande}. Namely, he characterized the set of discontinuity points of some approximately continuous function f with property
$(s_1)$ (see \cite{MarcinGrande}) in terms of so called $(a)$ condition:
For each set $U\in\tau_d\setminus\lbrace\emptyset\rbrace$ contained in the closure $\cl(A)$ of the set $A$ the
set $U \cap A$ is nowhere dense in $U$.
In 2003 Zbigniew Grande and Ewa Stronska proved that the family of all sets satisfying
the condition $(a)$ is an ideal of sets, which is $G_{\delta\sigma}$-generated, but not 
$F_{\sigma}$-generated 
(see [GS]). They also observed that every set satisfying condition $(a)$ is nowhere dense
and of Lebeque measure zero. Answering the question of Grande and Stronska,
in 2011 Frankowska and Nowik proved that ideal $(a)$ is not $G_{\delta}$-generated (\cite{FN2}).
Various further properties of the ideal $(a)$ were investigated in \cite{N}
%%%, see also [3].
where there was also defined a general version of the ideal for another
topologies. Namely, assume
that $(X, \tau_1,\tau_2)$ is a bitopological space 
(i.e. $\tau_1$ and $\tau_2$ are topologies on $X$, 
see \cite{D} or )
such that $\tau_1$ is coarser than $\tau_2$:
$\tau_1 \subseteq \tau_2$. Then 
define $\aideal(\tau_1, \tau_2) = \{ A \subseteq X \colon
\forall_{U \in \topWithoutEmptyset{\tau_2}}
\exists_{W \in \tau_1} U \cap W \not= \emptyset \wedge
U \cap W \cap A = \emptyset\}$.
It turned out that $\aideal(\tau_e, \tau_d)$ is exactly the ideal $(a)$
, where $\tau_e$ is the standard Euclidean topology and 
$\tau_d$ is the standard density topology (see \cite{N} Theorem 2.4).
In said article it was proven that the ideal $\aideal(\tau_1, \tau_2)$
is equivalent to the collection 
$\bigcap_{\tau_1 \subseteq \tau \subseteq \tau_2 \atop \tau \hbox{\tiny \it
topology}} \mathsf{ND}(\tau)$ (see Theorem 2.6, \cite{N}).
%%% TODO: czy tu nie powinno aby byc "The further(...)"
Further investigations of the properties of the ideal $(a)$ can be find in the paper  
\cite{FG}.
Notice also that we can consider the ideal $(a)$ defined on the Cantor
space $2^{\omega}$ instead the real line. 
Frankowska and Nowik in \cite{FN1} proved that also in that case
the ideal $\aideal$ on the Cantor space is 
not $G_{\delta}$-generated (in fact, they proved implicite that the ideal $(a)$ is not
$F_{\sigma\delta}$ generated).
  We try to sketch the history of the next class of sets. Namely, in \cite{GS}
the authors showed that for any closet set $E\subseteq \real$ we have
$E\setminus \Phi(E) \in (a)$, where $\Phi(E)$ denote a set of
density points of $E$. Since for every closed set $E$ we have
$\mathint_{\tau_d}(E) = \Phi(E)$ it is natural to consider for the
general case the family 
$\aidealprime = \lbrace A \subseteq X\colon \exists_{E \subseteq X} 
X\setminus E \in \tau_1 \wedge A \subseteq E \setminus  \mathint_{\tau_2}(E) \rbrace$
which is contained in $\aideal$ (see \cite{GS} and \cite{N}).

\section{Definitions and Notation}

For explain some topological notions we refer for example \cite{En}.
If $s \in \finsub$, $U \in \infsub$ and $\max (s) < \min(U)$,
then by $[s, U]$ we denote $\{ A \in \infsub: s \seq A \seq s \cup U\}$.
We call such a set an {\it Ellentuck set}. The collection of all
Ellentuck sets establish a base of a topology, called 
{\it Ellentuck topology}. We denote this topology 
by $\tauEllentuck$. This topology was 
(...)
%%%  We say that a set $X \seq \infsub$ is Ramsey null
%%%(or shortly, X is $CR_0$) iff for every $[s, U]$, there exists $[s, V] \seq [s, U]$ such that
%%%$[s, V] \cap X = \emptyset$.  If $X \seq \ca$, then we will call it a Ramsey null
%%%set iff for every $[s, U]$, there is $[s, V] \seq [s, U]$,
%%%so that  $\{ \chi(x): x \in [s, V]\} \cap X = \emptyset$, where for $x \in \infsub$, $\chi(x)$ denotes its characteristic function.

It is well known that the field defined via
Marczewski-Burstin representation:
$\lbrace A \subseteq \real\colon 
\forall_{U \in \tau_e \setminus \lbrace \emptyset \rbrace}
\exists_{V \in \tau_e \setminus \lbrace \emptyset \rbrace \atop V \subseteq U}
V \subseteq A \vee V \cap A = \emptyset\rbrace$
is the field of sets of form $U \bigtriangleup E$
where $U$ is open and $E$ is nowhere dense
which is the same as the collection of sets $A$ such that
$\mathrm{Fr}(A)$ is nowhere dense.
This leads to the following definition:

Let us define 
\begin{definition}
$\Afield(\tau_1, \tau_2) = \{ Z \subseteq X \colon
\forall_{U \in \tau_2 \setminus \{ \emptyset\}}
\exists_{W \in \tau_1} U \cap W \not= \emptyset \wedge
(U \cap W \cap Z = \emptyset \vee U \cap W \subseteq Z)\}$.
\end{definition}

Notice that $\Afield(\tau_1, \tau_2)$ is a field of sets.

\section{Results}

\begin{theorem}
\label{collapse-theorem}
%Denote by $\cK$ the following collection of subsets of the real line:
%$\lbrace A \subseteq \real \colon 
%\exists_{V\in\tau_e\setminus\lbrace\emptyset\rbrace}  
%\mathint (A \bigtriangleup V) = \emptyset\rbrace$.

Suppose that $\tau_1$ and $\tau_2$ are topologies on a set $X$ such
that $\tau_1 \subseteq \tau_2$.
The following conditions are equivalent:
\begin{enumerate}
\item $\tau_2 \cap \mathsf{ND}(\tau_1) = \lbrace \emptyset \rbrace$;
\item $\aideal(\tau_1, \tau_2) = \mathsf{ND}(\tau_1)$;
\item $\mathsf{ND}(\tau_1) \subseteq \mathsf{ND}(\tau_2)$.
\end{enumerate}
\end{theorem}
									
\begin{proof}  
%Notice that although the collection of sets with an empty interior is not an ideal,
%we have that $\cK = \lbrace (V \setminus B_1) \cup B_2\colon
%\mathint(B_1) = \mathint(B_2) = \emptyset, 
%V\in\topWithoutEmptyset{\tau_e}\rbrace$.
%%%TODO: czy jednak nie przydaloby sie tutaj jakies uzasadnienie?
$(1) \implies (2)$

It suffices to show the inclusion $\subseteq$, since
$\aideal(\tau_1, \tau_2) \subseteq \mathsf{ND}(\tau_1)$.
Suppose $A\in \mathsf{ND}(\tau_1)$ and choose
$W \in \tau_2\setminus\lbrace \emptyset \rbrace$.
By the assumption $(1)$ we have $W \not\in \mathsf{ND}(\tau_1)$, 
hence $\mathint_{\tau_1} \cl_{\tau_1} (W) \not= \emptyset$.
Since $A \in \mathsf{ND}(\tau_1)$ there exists 
$V\in \tau_1\setminus\lbrace\emptyset\rbrace$ such 
that $V \subseteq \mathint_{\tau_1} \cl_{\tau_1} (W) \setminus A$. 
Since $\emptyset \not= V \subseteq \cl_{\tau_1} (W)$, 
we have $W \cap V \not= \emptyset$. Moreover,
$W \cap V \cap A = \emptyset$. Therefore 
$A \in \aideal(\tau_1, \tau_2)$.

%Suppose $A \in \mathsf{ND}(\real)$ and let $W \in \tau_2\setminus\lbrace \emptyset \rbrace$.
%Then there exists $V\in\topWithoutEmptyset{\tau_e}$ such that
%$\mathint (W \bigtriangleup V) = \emptyset$. Since $A$ is nowhere dense, there
%exists $U \in \topWithoutEmptyset{\tau_e}$ such that
%$U \subseteq V$ and $U \cap A = \emptyset$. Then
%$\mathint(U \setminus W) \subseteq \mathint(V \setminus W) = \emptyset$,
%hence $U \cap W \not= \emptyset$. Thus $A \in \aideal(\tau_e, \tau_2)$.
%On the other hand, we always have 
%$\aideal(\tau_e, \tau_2) \subseteq \mathsf{ND}(\tau_e) = \mathsf{ND}(\real)$,
$(2) \implies (3)$.
It follows easily, since 
$\aideal(\tau_1, \tau_2) \subseteq \mathsf{ND}(\tau_2)$.

$\neg(1)\implies \neg(3)$.
Suppose that $W\in\tau_2\cap\mathsf{ND}(\tau_1) \setminus \lbrace\emptyset\rbrace$.
Obviously $W\not\in\mathsf{ND}(\tau_2)$ thus
$W \in \mathsf{ND}(\tau_1) \setminus \mathsf{ND}(\tau_2)$,
which finishes the proof.
\end{proof}

%Moreover, let us notice that the collection $\cK$ 
%is the collection of sets $A$ such that $\mathint(\cl(A)) \not=\emptyset$. 
%Indeed, we have that $\mathint(A \setminus \mathint(\cl(A))) = \emptyset$
%and 
%$\mathint(\mathint(\cl(A)) \setminus A) = \emptyset$, so 
%if $\mathint(\cl(A)) \not= \emptyset$ then $A\in \cK$.
%On the other hand, suppose that $A \in \cK$, so
%there exists $B_1, B_2 \subseteq \real$ and 
%$V\in\topWithoutEmptyset{\tau}$
%such that 
%$\mathint(B_1) = \mathint(B_2) = \emptyset$
%and  
%$A = (V \setminus B_1) \cup B_2$. 
%Then 
%$\mathint(\cl(A)) \supseteq 
%\mathint(\cl(V \setminus B_1)) = \mathint(\cl(V)) \supseteq V \not= \emptyset$.
%%%%%%%%%%%%%%%%%%%%%%%%%%%%%%%%%%%%%%%%%%%%
\begin{corollary}
\label{corollary:category-density-topology}
If $\cF_{\meager}$ is the category analogue of the density topology (see \cite{PWBW}) 
then 
$(a)(\tau_e, \cF_{\meager}) = \mathsf{ND}(\real)$.
\end{corollary}

\begin{proof}
This follows from the fact that 
$\cF_{\meager} \setminus\lbrace\emptyset\rbrace\subseteq \Baire\setminus\meager$
(see \cite{PWBW}, Remark 2) and
from Theorem \ref{collapse-theorem}
\end{proof}

Let us formulate a ''dual'' theorem:

\begin{theorem}
Suppose that $\tau_1$ and $\tau_2$ are topologies on a set $X$ such
that $\tau_1 \subseteq \tau_2$.
Then we have implications $(1) \implies (2) \implies (3)$
where:
\begin{enumerate}
\item $\aideal(\tau_1, \tau_2) = \mathsf{ND}(\tau_2)$;
\item $\mathsf{ND}(\tau_2) \subseteq \mathsf{ND}(\tau_1)$.
\item $\tau_1 \cap \mathsf{ND}(\tau_2) = \lbrace \emptyset \rbrace$;
\end{enumerate}
\end{theorem}

\begin{proof}
$(1)\implies (2)$ is obvious, since
$\aideal(\tau_1, \tau_2) \subseteq \mathsf{ND}(\tau_1)$.

$\neg(3)\implies \neg(2)$.
Suppose that $W\in\tau_1\cap\mathsf{ND}(\tau_2) \setminus \lbrace\emptyset\rbrace$.
Obviously $W\not\in\mathsf{ND}(\tau_1)$ thus
$W \in \mathsf{ND}(\tau_2) \setminus \mathsf{ND}(\tau_1)$,
which finishes the proof.
\end{proof}

Notice that in the previous theorem we don't need to have
the implication $(3) \implies (1)$. The counterexample 
is the density topology ideal $\aideal$, for $\tau_1 = \tau_e$
and $\tau_2 = \tau_d$. Indeed, we have 
that $\mathsf{ND}(\tau_2)$ is a collection of Lebesgue null sets, 
so $\tau_1 \cap \mathsf{ND}(\tau_2) = \lbrace \emptyset \rbrace$,
but of course neither $\mathsf{ND}(\tau_2) \subseteq \mathsf{ND}(\tau_1)$ 
nor $\aideal(\tau_1, \tau_2) = \mathsf{ND}(\tau_2)$.

It is easy to see that in the definition of $\aideal(\tau_1, \tau_2)$
(see Section \ref{section:preliminaries})
we may assume that instead of taking $W \in \tau_1$
we can take $W$ from any fixed base $\cB$ of the topology
$\tau_1$. However, we cannot assume that (analogously)
the set $U$ can be taken only from (any fixed) base of
the topology $\tau_2$. The counterexample is based 
on Example \ref{example:ellentuck}

\subsection{Examples}
\begin{example}
$X = \real$,
$\tau_1 = \tau_e$ and
$\tau_2 = \cF_{\meager}$
where $\cF_{\meager}$ is the category analogue of the density topology (see \cite{PWBW}).
We argued (see Corollary \ref{corollary:category-density-topology}) that we have 
$(a)(\tau_e, \cF_{\meager}) = \mathsf{ND}(\real)$.
\end{example}
\begin{example}
\label{example:ellentuck}
$X = [\omega]^\omega$,
$\tau_1$ is a standard topology (on the Baire space of infinite natural sequences) and
$\tau_2 = \tauEllentuck$, where $\tauEllentuck$ is the Ellentuck topology.
\end{example}

\subsection{$\aideal$ zdefiniowany via Marczewski - Burstin}

Assume that $\cF_1$ and $\cF_2$ are MB - families of nonempty 
subsets of $\real$ such that:
\begin{enumerate}	
\item
\label{mb-condition-subset}
  $\cF_1 \subseteq \cF_2$
\item 
\label{mb-condition-f1}
  $\forall_{E,F \in \cF_1} E \cap F \not= \emptyset \implies E \cap F \in \cF_1$
\item
\label{mb-condition-f1f2}
  $\forall_{E \in cF_1} \forall_{W \in \cF_2} E \cap W \not= \emptyset \implies E \cap W \in \cF_2$  
\end{enumerate}	

%%%TODO: a moze lepiej by brzmialo: analogously like...?
Analogously to the notion of bitopological spaces we call a structure
$\langle \cF_1, \cF_2 \rangle$ a {\it\biMB} system.

\begin{theorem}
Suppose that $\langle \cF_1, \cF_2 \rangle$ is a \biMB system.
Then the collection
$\aideal(\cF_1, \cF_2) = \lbrace A \subseteq \real\colon 
\forall_{W\in \cF_2} \exists_{U \in \cF_1}
W \cap U \not= \emptyset \wedge W \cap U \cap A = \emptyset
\rbrace$ 
is an ideal. 
\end{theorem}
\begin{proof}
It suffices only to check that if $A, B \in \aideal(\cF_1, \cF_2)$ then
$A\cup B \in \aideal(\cF_1, \cF_2)$.
So suppose that $W \in \cF_2$, then there exists $U\in\cF_1$
such that 
$W \cap U \not= \emptyset$ and $W \cap U \cap A = \emptyset$.
By condition \ref{mb-condition-f1f2} 
we know that $W \cap U \in \cF_2$ so
we conclude that
there exists a set $V \in \cF_1$ such that
$W \cap U \cap V \not= \emptyset$ and $W \cap U \cap V \cap A = \emptyset$.
Since by condition \ref{mb-condition-f1} we know that 
$U \cap V \in \cF_1$, this ends the proof.
\end{proof}


%%%%%%%%%%%%%%%%%%%%%%%%%%%%%%%%%%%%%%%%%%%%%%%%%%%%%%%%%%%%%%%%%%%%%%%%%%%%%
%                        The Bibliography                                   %
%%%%%%%%%%%%%%%%%%%%%%%%%%%%%%%%%%%%%%%%%%%%%%%%%%%%%%%%%%%%%%%%%%%%%%%%%%%%%
\begin{thebibliography}{10}
\bibitem[BT]{BT}
  M.K. Bose, R. Tiwari, \emph{On increasing sequences of topologies
on a set}, Riv. Mat. Univ. Parma, \textbf{7} (7), (2007), 173--183.

\bibitem[D]{D}
  B. P. Dvalishvili, \emph{Bitopological Spaces: Theory, 
Relations with Generalized Algebraic Structures, and Applications}.
North-Holland Mathematics Studies, \textbf{199}. 
Elsevier Science B.V., Amsterdam, 2005.

\bibitem{MarcinGrande} M.~Grande, \emph{On the sets of discontinuity points of functions satisfying some approximate
quasi- continuity conditions}, Real Anal. Exchange \textbf{27(2)} (2001-2002), 773--781.

\bibitem[En]{En}
R.Engelking \emph{General Topology}, Heldermann Verlag, Berlin, 1989.

\bibitem[FG]{FG}
M. Frankowska, S.G\l{}\c{a}b, 
\emph{On some $\sigma$-ideal without ccc}, Colloquium Mathematicum \textbf{158 (1)} (2019), 127--140.

\bibitem[FN1]{FN1} M. Frankowska, A. Nowik, \emph{On some ideal defined by density topology in the Cantor set},
Georgian Math. J. \textbf{19, No. 1}, 93-99 (2012).

\bibitem[FN2]{FN2} M. Frankowska, A. Nowik, \emph{The ideal (a) is not $G_\delta$ generated}, 
Coll. Math. vol. \textbf{125, no.1} (2011), 7-14.

\bibitem[G]{G}
  M.Grande {\em On the measurability of functions satisfying
some approximate quasicontinuity conditions},
{Real Analysis Exchange} {30 (2004/05)}, no. 1, 1 -- 10.

\bibitem[Gl]{Gl}
S.G\l{}\c{a}b
{\em Descriptive properties related to porosity and density for compact sets on the real line,}
Acta Math. Hungar., 116 (1-2) (2007), 61--71.

\bibitem[GS]{GS}
  Z.Grande, E.Stro\'nska, \emph{On an ideal of linear sets},
Demonstratio Mathematica, Vol. 36, No 2, (2003), 
307 -- 311.

\bibitem[Kelly]{Kelly}
J.C.Kelly, \emph{Bitopological spaces.} Proc. Lond. Math. Soc. 
\textbf{(3) 13} (1963), 71--89.

\bibitem[Ke]{Ke}
{A. S. Kechris}, {\em Classical Descriptive Set Theory}, Springer, Berlin,
1995. 

\bibitem[N]{N}
  A. Nowik, \emph{Notes on the ideal $(a)$},
{Tatra Mountains Mathematical Publications,} 46 (2010), 41--45.

\bibitem[Pl]{Pl} S.Plewik, \emph{On completely Ramsey sets}, 
Fund. Math. \textbf{127} (1986), 127-132.

\bibitem[PWBW]{PWBW}
W. Poreda, E. Wagner-Bojakowska, W\l{}adys\l{}aw Wilczy\'{n}ski
{\em A category analogue of the density topology}
Fundamenta Mathematicae 125 (1985), 167--173.

\bibitem[T]{T}
 F.D.Tall, {\em The density topology,}
{Pacific J. Math.,} Vol 62, Number 1 (1976), 275--284
\end{thebibliography}

\begin{center}
\ctimenow
\end{center}

\end{document}