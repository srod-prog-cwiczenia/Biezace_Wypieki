%%%%%%%%%%%%%%%%%%%%%%%%%%%%%%%%%%%%%%%%%%%%%%%%%%%%%%%%%%%%%%%%%%%%%%%%%
%                                                                       %
%                  praca o ideale (a)                                   %
%                                                                       %
%%%%%%%%%%%%%%%%%%%%%%%%%%%%%%%%%%%%%%%%%%%%%%%%%%%%%%%%%%%%%%%%%%%%%%%%%
%\documentclass[12pt]{article}
%%% aby uzyc amsart w pelnej krasie:
% 1. odkomentowac \address i \email
% 2. odkomentowac \keywords{} \subjclass
\documentclass[12pt]{amsart}
\usepackage{amssymb}
\usepackage{amsmath}
\usepackage{latexsym}
\usepackage{amsfonts}
\usepackage{enumerate}
\usepackage{mathtools}
%\usepackage[mathscr]{eucal}
\usepackage[mathscr]{euscript}
\usepackage{eqlist}
\usepackage{amsthm}
%%%%%%%%%%%%%%% Polish letter packages %%%%%%%%%%%%%%%%%%%%%%%%%%%%%%%%
\usepackage[polish]{babel}
\usepackage[utf8]{inputenc}
\usepackage{t1enc}
%%% pakiet do generowania belkotu (wypelniacza):
\usepackage{lipsum}
%%% pakiet do dodawania fikusnych notek `a la ``todo''
\usepackage{todonotes}
%---------------------------------------------------------------------
% Theorems 
%---------------------------------------------------------------------
 
% Theorem style 'plain' are for: Theorem, Lemma, Corollary,
% Proposition, Conjecture, Criterion, Algorithm
\theoremstyle{plain}
\newtheorem{theorem}{Theorem}[section]
\newtheorem{lemma}[theorem]{Lemat}
\newtheorem{corollary}[theorem]{Corollary}
\newtheorem{conclusion}[theorem]{Corollary}
\newtheorem{claim}[theorem]{Claim}
\newtheorem{fact}[theorem]{Fact}
\newtheorem{proposition}[theorem]{Proposition}
\newtheorem{axiom}{Axiom}
% Theorem style 'definition' are for: Definition, Condition, Problem,
% Example
\theoremstyle{definition}
\newtheorem{definition}[theorem]{Definition}
\newtheorem{example}[theorem]{Example}
\newtheorem{exercise}{Exercise}
\newtheorem*{solution}{Solution}
\newtheorem{remark}[theorem]{Remark}
\newtheorem{Problem}[theorem]{Problem}
% Theorem style 'remark' are for: Remark, Note, Notation, Claim,
% Summary, Acknowledgement, Case, Conclusion
\theoremstyle{remark}
\newtheorem*{notation}{Notation}
\newtheorem*{acknowledgement}{Acknowledgement}

%%%%%% makra:

%A
\newcommand{\afc}{AFC}
\newcommand{\afcbar}{\overline{AFC}}
\newcommand{\arr}{\rightarrow}
\newcommand{\Arr}{\Rightarrow}

%B
\newcommand{\baire}{\omega^{\omega}}
\newcommand{\Baire}{\mathfrak{Baire}}
\newcommand{\Bor}{\mbox{${\mathcal B}or$}}
%Previous seems to be much finer than next.
%\newcommand{\Bor}{{\it Bor}}
\newcommand{\borelucrz}{Borel-UCR_0}

%C
\newcommand{\ca}{2^{\omega}}
\newcommand{\cantor}{\ca}
\newcommand{\Card}[1]{\Vert #1 \Vert}
\newcommand{\cl}{\mathit{cl}}

%D
\newcommand{\dom}{{\rm dom}}
\newcommand{\dummy}{{\tt Blah blah blah}}

%E
\newcommand{\Even}{\hbox{\rm \tiny Even}}

%F
\newcommand{\finsub}{[\omega]^{<\omega}}
\newcommand{\forces}{\mathrel{\|}\joinrel\mathrel{-}}

%G
\newcommand{\Graph}{\hbox{\it Graph}}

%H
\newcommand{\homeomorphic}{\approx}

%I
\newcommand{\incr}{\omega^{\uparrow \omega }}
\newcommand{\infsub}{[\omega]^{\omega}}

%L
\newcommand{\la}{\langle}

%M
\newcommand{\meager}{{\mathcal{MGR}}}
\newcommand{\minideal}{${\cal F}_{\hbox{\rm \scriptsize min}}(\neg
  D)\;$}

%N
\newcommand{\neglig}{{\cal N}}
\newcommand{\nnatural}{\mathbb{N}}

%O
\newcommand{\Odd}{\hbox{\rm \tiny Odd}}

%P
\newcommand{\Part}{{\it Part}}
\newcommand{\Perf}{{\it Perf}}
%%%\newcommand{\proof}{\flushleft{ \sc Proof. } \\ }
\newcommand{\Proof}[1]{\bigbreak\noindent{\bf Proof #1}\enspace}

%Q
%%%\newcommand{\qed}{{\hfill\vrule height6pt width6pt depth1pt\medskip}}
%\newcommand{\qed}{\sharp}
\newcommand{\QED}{\hspace{0.1in} \Box \vspace{0.1in}}

%R
\newcommand{\ra}{\rangle}
\newcommand{\ran}{{\rm ran}}
\newcommand{\rational}{\mathbb{Q}}
\newcommand{\real}{\mathbb{R}}

%S
\newcommand{\seq}{\subseteq}
%%%\newcommand{\square}{\hbox{\ \ \ \ \ \vrule\vbox{\hrule\phantom{o}\hrule}\vrule}}
% a small restriction:
\newcommand{\srestriction}{{\hbox{${\scriptstyle\,|\grave{}\,}$}}}
\newcommand{\supp}{\mathit{supp}}

%U
\newcommand{\up}{\uparrow}
\newcommand{\ucrz}{UCR_0}

%%%%%%%%%%%%%%%%%%%%%% Calligraphic font commands %%%%%%%%%%%%%%%%%%%%%%%%%%
\newcommand{\cA}{{\mathcal A}}
\newcommand{\cB}{{\mathcal B}}
\newcommand{\cC}{{\mathcal C}}
\newcommand{\cD}{{\mathcal D}}
\newcommand{\cE}{{\mathcal E}}
\newcommand{\cF}{{\mathcal F}}
\newcommand{\cG}{{\mathcal G}}
\newcommand{\cH}{{\mathcal H}}
\newcommand{\cI}{{\mathcal I}}
\newcommand{\cJ}{{\mathcal J}}
\newcommand{\cK}{{\mathcal K}}
\newcommand{\cL}{{\mathcal L}}
\newcommand{\cM}{{\mathcal M}}
\newcommand{\cN}{{\mathcal N}}
\newcommand{\cO}{{\mathcal O}}
\newcommand{\cP}{{\mathcal P}}
\newcommand{\cQ}{{\mathcal Q}}
\newcommand{\cR}{{\mathcal R}}
\newcommand{\cS}{{\mathcal S}}
\newcommand{\cT}{{\mathcal T}}
\newcommand{\cU}{{\mathcal U}}
\newcommand{\cV}{{\mathcal V}}
\newcommand{\cW}{{\mathcal W}}
\newcommand{\cX}{{\mathcal X}}
\newcommand{\cY}{{\mathcal Y}}
\newcommand{\cZ}{{\mathcal Z}}
%%%%%%%%%%%%%%%%%%%%%% inne %%%%%%%%%%%%%%%%%%%%%%%%%%%%%%%%%%%%%%%
\newcommand{\SqrFr}{\mathbb{SF}}
\newcommand{\Primes}{\mathit{Primes}}
\newcommand{\mathint}{\mathit{int}}
\newcommand{\mathcl}{\mathit{cl}}
\newcommand{\exampleGFnotK}{\mathit{Ex}}
%%%%%%%%%%%%%%%%%%%%%% Some Greek fonts %%%%%%%%%%%%%%%%%%%%%%%%%%%%%%%%%%%%%%%
\newcommand{\oo}{\omega}
\newcommand{\bb}{\beta}
\newcommand{\dd}{\delta}
\newcommand{\ee}{\varepsilon}
\newcommand{\kk}{\kappa}
%%%\newcommand{\th}{\theta}
%%%% makra specyficzne dla tej pracy:
\newcommand{\aideal}{\mathit{(a)}}
\newcommand{\aidealprime}{\mathit{(a^\prime)}}
\newcommand{\Afield}{\mathit{(A)}}
\newcommand{\topWithoutEmptyset}[1]{#1\setminus\lbrace\emptyset\rbrace}
\newcommand{\ND}{\mathsf{ND}}
\newcommand{\biMB}{bi-Marczewski-Burstin}
\newcommand{\tauEllentuck}{\tau_{\mathrm{EL}}}
\newcommand{\baseEllentuck}{\cB_{\mathrm{EL}}}
\newcommand{\ninomega}{n\in\omega}
\newcommand{\Hereditary}{\mathcal{H}}
\newcommand{\FatPerf}{\mathit{FatPerf}}
%\renewcommand\abstractname{Summary}
%%%To print the date and time on each page
%%% comment out if not needed (next 14 lines)
\makeatletter
{\newcount\@hour}
{\newcount\@minute}
\def\timenow{\@hour=\time \divide\@hour by 60
\number\@hour:
  \multiply\@hour by 60 \@minute=\time
  \global\advance\@minute by -\@hour
  \ifnum\@minute<10 0\number\@minute\else
  \number\@minute\fi}
\def\ctimenow{\hfil{\tt \jobname.tex, \today~Time: \timenow }\hfil}
      \let\@oddfoot\ctimenow\let\@evenfoot\ctimenow
\makeatother

\pagestyle{myheadings}
\markboth{{\bf Ideał generowany przez $\aprimeideal$ nie jest równy $\aideal$}}{\bf \today}

% To see corrections comment next line and uncomment the second one
\newcommand{\correction}[2]{#1}
% \newcommand{\correction}[2]{#2}

\pagestyle{myheadings}
% To see corrections comment next line and uncomment the second one

%\author{}
%\address[]{}

\begin{document}
\title{Ideał generowany przez $\aidealprime$ nie jest równy ideałowi $\aideal$}

%\begin{abstract}
%(...)
%\end{abstract}

\maketitle

\section{Garść oznaczeń i definicji}
$\Phi(E)$ oznacza zbiór punktów gęstości zbioru $E$.
Jeśli $f \colon X \to Y$ jest funkcją to przez
$f[A]$ oznaczamy obraz zbioru $A\subseteq X$, 
zaś $f^{-1}[B]$ przeciwobraz zbioru $B \subseteq Y$.
$\tau_d$ oznacza topologię gęstości
(czyli rodzinę zbiorów otwartych w tej topologii),
to znaczy rodzinę takich zbiorów mierzalnych $A\subseteq\real$,
że $A \subseteq \Phi(A)$. Przez $\tau_e$ oznaczamy rodzinę
zbiorów otwartych w standardowej topologii euklidesowej na $\real$.

\subsection{Nieodzowne pojęcia używane w tej notatce}
Ideał $\aideal$ posiada wiele równoważnych charakteryzacji.
Tu będziemy stale korzystać z następującej:
\begin{definition}
  Zbiór $X \subseteq \real$ jest w ideale $\aideal$ wtedy i tylko
  wtedy gdy spełniony jest następujący warunek:
  \[
  \forall_{W \in \tau_d\setminus\lbrace\emptyset\rbrace}
    \exists_{U \in \tau_e} W \cap U \not= \emptyset \wedge
    W \cap U \cap X = \emptyset.
  \]
\end{definition}

Przez rodzinę $\aidealprime$ rozumiemy rodzinę takich zbiorów
$X\subseteq\real$ dla których istnieje zbiór
domknięty (w standardowej topologii) $E \subseteq\real$
taki, że $X \subseteq \Phi(E) \setminus E$. Wiadomo
że $\aidealprime\subseteq\aideal$ oraz że sigma ideały
generowane przez rodziny $\aidealprime$ oraz $\aideal$
są identyczne (\cite{FG}, Theorem 12).
%%%%%%%%%%%%%%%%%%%%%%%%%%%%%%%%%%%%%%%%%%%%%%%%%%%%%%%%%%%%%%%%%%%%%%%%%%%%%
%                        The Bibliography                                   %
%%%%%%%%%%%%%%%%%%%%%%%%%%%%%%%%%%%%%%%%%%%%%%%%%%%%%%%%%%%%%%%%%%%%%%%%%%%%%
\begin{thebibliography}{10}

\bibitem[FG]{FG}
M. Frankowska, S.G\l{}\c{a}b, 
\emph{On some $\sigma$-ideal without ccc}, Colloquium Mathematicum \textbf{158 (1)} (2019), 127--140.
\end{thebibliography}

\end{document}

%%% ten fragment zostal usuniety z konstrukcji przykladu a versus a'
%%% ten fragment dowodu dotyczy zbioru A2 i jest
%%% w zasadzie nadmiarowy bo wystarczy rozwazyc przypadek zbioru A_{n+1}.
%%%%%%%%%%%%%%%%%%%%%%%%%%%%%%%%%%%%%%%%%%%%%
%Definiujemy:
%\[ A_2 = \bigcup_{m=1}^{\infty} l_m[A_1] \cup \lbrace p^* \rbrace.
%\]
%
%\flushleft{\it Fakt 1:} $A_2 \in \aideal$.
%\begin{proof}
%Wiemy że $l_m[A_1] \in \aideal$, bo zarówno odwzorowanie liniowe $l_m$
%jak i $l_m^{-1}$ są homeomorfizmami które zachowują punkty gęstości.
%Połóżmy zatem w Lemacie \ref{lemat-o-sumie} 
%$B_m = l_m[A_1]$ i z faktu iż $\aideal$ jest ideałem
%zawierającym singletony dostajemy natychmiast wniosek
%że $A_2 \in \aideal$.
%
%\flushleft{\it Fakt 2:} $A_2$ nie jest sumą dwóch zbiorów z rodziny $\aidealprime$.
%Przypuśćmy że $A_2 \subseteq (E_1 \setminus \Phi(E_1)) \cup (E_2 \setminus \Phi(E_2))$
%dla pewnych dwóch zbiorów domkniętych $E_1, E_2$.  Przypuśćmy że dla pewnego 
%$m = 1,2,\ldots$ mielibyśmy że $l_m[A_1] \cap E_2 = \emptyset$.
%Wówczas $l_m[A_1] \subseteq E_1 \setminus \Phi(E_1)$, czyli 
%$A_1 \subseteq l_m^{-1}[E_1] \setminus \Phi(l_m^{-1}[E_1])$, gdyż
%odwzorowanie liniowe $l_m$ zachowuje punkty gęstości (i tak samo odwzorowanie odwrotne). Ponieważ 
%przeprowadza ono też zbiory domknięte na zbiory domknięte (i tak samo odwzorowanie odwrotne)
%więc daje to sprzeczość z tym że $A_1 \not\in \aidealprime$.
%Zatem $l_m[A_1] \cap E_2 \not= \emptyset$.
%Identyczne rozumowanie (ze względu na symetrię założeń o $E_1$ i $E_2$)
%pokazuje że także i $l_m[A_1] \cap E_1 \not= \emptyset$.
%Ponieważ $l_m[A_1] \subseteq (a_m, b_m)$
%więc $P \subseteq E_1$ i $P \subseteq E_2$. 
%Mamy zatem $p^* \in \Phi(P) \subseteq \Phi(E_1)$
%i $p^* \in \Phi(P) \subseteq \Phi(E_2)$
%czyli $p^* \not\in (E_1 \setminus \Phi(E_1)) \cup (E_2 \setminus \Phi(E_2))$
%co jest niemożliwe bo $p^* \in A_2$.
%\end{proof}
%%%%%%%%%%%%%%%%%%%%%%%%%%%%%%%%%%%%%%%%%
