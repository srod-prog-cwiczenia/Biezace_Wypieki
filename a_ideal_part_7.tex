%%%%%%%%%%%%%%%%%%%%%%%%%%%%%%%%%%%%%%%%%%%%%%%%%%%%%%%%%%%%%%%%%%%%%%%%%
%                                                                       %
%                  praca o ideale (a)                                   %
%                                                                       %
%%%%%%%%%%%%%%%%%%%%%%%%%%%%%%%%%%%%%%%%%%%%%%%%%%%%%%%%%%%%%%%%%%%%%%%%%
%\documentclass[12pt]{article}
%%% aby uzyc amsart w pelnej krasie:
% 1. odkomentowac \address i \email
% 2. odkomentowac \keywords{} \subjclass
\documentclass[12pt]{amsart}
\usepackage{amssymb}
\usepackage{amsmath}
\usepackage{latexsym}
\usepackage{amsfonts}
\usepackage{enumerate}
\usepackage{mathtools}
%\usepackage[mathscr]{eucal}
\usepackage[mathscr]{euscript}
\usepackage{eqlist}
\usepackage{amsthm}
%%%%%%%%%%%%%%% Polish letter packages %%%%%%%%%%%%%%%%%%%%%%%%%%%%%%%%
\usepackage[polish]{babel}
\usepackage[utf8]{inputenc}
\usepackage{t1enc}
%%% pakiet do generowania belkotu (wypelniacza):
\usepackage{lipsum}
%%% pakiet do dodawania fikusnych notek `a la ``todo''
\usepackage{todonotes}
%---------------------------------------------------------------------
% Theorems 
%---------------------------------------------------------------------
 
% Theorem style 'plain' are for: Theorem, Lemma, Corollary,
% Proposition, Conjecture, Criterion, Algorithm
\theoremstyle{plain}
\newtheorem{theorem}{Theorem}[section]
\newtheorem{lemma}[theorem]{Lemat}
\newtheorem{corollary}[theorem]{Corollary}
\newtheorem{conclusion}[theorem]{Corollary}
\newtheorem{claim}[theorem]{Claim}
\newtheorem{fact}[theorem]{Fact}
\newtheorem{proposition}[theorem]{Proposition}
\newtheorem{axiom}{Axiom}
% Theorem style 'definition' are for: Definition, Condition, Problem,
% Example
\theoremstyle{definition}
\newtheorem{definition}[theorem]{Definition}
\newtheorem{example}[theorem]{Example}
\newtheorem{exercise}{Exercise}
\newtheorem*{solution}{Solution}
\newtheorem{remark}[theorem]{Remark}
\newtheorem{Problem}[theorem]{Problem}
% Theorem style 'remark' are for: Remark, Note, Notation, Claim,
% Summary, Acknowledgement, Case, Conclusion
\theoremstyle{remark}
\newtheorem*{notation}{Notation}
\newtheorem*{acknowledgement}{Acknowledgement}

%%%%%% makra:

%A
\newcommand{\afc}{AFC}
\newcommand{\afcbar}{\overline{AFC}}
\newcommand{\arr}{\rightarrow}
\newcommand{\Arr}{\Rightarrow}

%B
\newcommand{\baire}{\omega^{\omega}}
\newcommand{\Baire}{\mathfrak{Baire}}
\newcommand{\Bor}{\mbox{${\mathcal B}or$}}
%Previous seems to be much finer than next.
%\newcommand{\Bor}{{\it Bor}}
\newcommand{\borelucrz}{Borel-UCR_0}

%C
\newcommand{\ca}{2^{\omega}}
\newcommand{\cantor}{\ca}
\newcommand{\Card}[1]{\Vert #1 \Vert}
\newcommand{\cl}{\mathit{cl}}

%D
\newcommand{\dom}{{\rm dom}}
\newcommand{\dummy}{{\tt Blah blah blah}}

%E
\newcommand{\Even}{\hbox{\rm \tiny Even}}

%F
\newcommand{\finsub}{[\omega]^{<\omega}}
\newcommand{\forces}{\mathrel{\|}\joinrel\mathrel{-}}

%G
\newcommand{\Graph}{\hbox{\it Graph}}

%H
\newcommand{\homeomorphic}{\approx}

%I
\newcommand{\incr}{\omega^{\uparrow \omega }}
\newcommand{\infsub}{[\omega]^{\omega}}

%L
\newcommand{\la}{\langle}

%M
\newcommand{\meager}{{\mathcal{MGR}}}
\newcommand{\minideal}{${\cal F}_{\hbox{\rm \scriptsize min}}(\neg
  D)\;$}

%N
\newcommand{\neglig}{{\cal N}}
\newcommand{\nnatural}{\mathbb{N}}

%O
\newcommand{\Odd}{\hbox{\rm \tiny Odd}}

%P
\newcommand{\Part}{{\it Part}}
\newcommand{\Perf}{{\it Perf}}
%%%\newcommand{\proof}{\flushleft{ \sc Proof. } \\ }
\newcommand{\Proof}[1]{\bigbreak\noindent{\bf Proof #1}\enspace}

%Q
%%%\newcommand{\qed}{{\hfill\vrule height6pt width6pt depth1pt\medskip}}
%\newcommand{\qed}{\sharp}
\newcommand{\QED}{\hspace{0.1in} \Box \vspace{0.1in}}

%R
\newcommand{\ra}{\rangle}
\newcommand{\ran}{{\rm ran}}
\newcommand{\rational}{\mathbb{Q}}
\newcommand{\real}{\mathbb{R}}

%S
\newcommand{\seq}{\subseteq}
%%%\newcommand{\square}{\hbox{\ \ \ \ \ \vrule\vbox{\hrule\phantom{o}\hrule}\vrule}}
% a small restriction:
\newcommand{\srestriction}{{\hbox{${\scriptstyle\,|\grave{}\,}$}}}
\newcommand{\supp}{\mathit{supp}}

%U
\newcommand{\up}{\uparrow}
\newcommand{\ucrz}{UCR_0}

%%%%%%%%%%%%%%%%%%%%%% Calligraphic font commands %%%%%%%%%%%%%%%%%%%%%%%%%%
\newcommand{\cA}{{\mathcal A}}
\newcommand{\cB}{{\mathcal B}}
\newcommand{\cC}{{\mathcal C}}
\newcommand{\cD}{{\mathcal D}}
\newcommand{\cE}{{\mathcal E}}
\newcommand{\cF}{{\mathcal F}}
\newcommand{\cG}{{\mathcal G}}
\newcommand{\cH}{{\mathcal H}}
\newcommand{\cI}{{\mathcal I}}
\newcommand{\cJ}{{\mathcal J}}
\newcommand{\cK}{{\mathcal K}}
\newcommand{\cL}{{\mathcal L}}
\newcommand{\cM}{{\mathcal M}}
\newcommand{\cN}{{\mathcal N}}
\newcommand{\cO}{{\mathcal O}}
\newcommand{\cP}{{\mathcal P}}
\newcommand{\cQ}{{\mathcal Q}}
\newcommand{\cR}{{\mathcal R}}
\newcommand{\cS}{{\mathcal S}}
\newcommand{\cT}{{\mathcal T}}
\newcommand{\cU}{{\mathcal U}}
\newcommand{\cV}{{\mathcal V}}
\newcommand{\cW}{{\mathcal W}}
\newcommand{\cX}{{\mathcal X}}
\newcommand{\cY}{{\mathcal Y}}
\newcommand{\cZ}{{\mathcal Z}}
%%%%%%%%%%%%%%%%%%%%%% inne %%%%%%%%%%%%%%%%%%%%%%%%%%%%%%%%%%%%%%%
\newcommand{\SqrFr}{\mathbb{SF}}
\newcommand{\Primes}{\mathit{Primes}}
\newcommand{\mathint}{\mathit{int}}
\newcommand{\mathcl}{\mathit{cl}}
\newcommand{\exampleGFnotK}{\mathit{Ex}}
%%%%%%%%%%%%%%%%%%%%%% Some Greek fonts %%%%%%%%%%%%%%%%%%%%%%%%%%%%%%%%%%%%%%%
\newcommand{\oo}{\omega}
\newcommand{\bb}{\beta}
\newcommand{\dd}{\delta}
\newcommand{\ee}{\varepsilon}
\newcommand{\kk}{\kappa}
%%%\newcommand{\th}{\theta}
%%%% makra specyficzne dla tej pracy:
\newcommand{\aideal}{\mathit{(a)}}
\newcommand{\aidealprime}{\mathit{(a^\prime)}}
\newcommand{\Afield}{\mathit{(A)}}
\newcommand{\topWithoutEmptyset}[1]{#1\setminus\lbrace\emptyset\rbrace}
\newcommand{\ND}{\mathsf{ND}}
\newcommand{\biMB}{bi-Marczewski-Burstin}
\newcommand{\tauEllentuck}{\tau_{\mathrm{EL}}}
\newcommand{\baseEllentuck}{\cB_{\mathrm{EL}}}
\newcommand{\ninomega}{n\in\omega}
\newcommand{\Hereditary}{\mathcal{H}}
\newcommand{\FatPerf}{\mathit{FatPerf}}
%\renewcommand\abstractname{Summary}
%%%To print the date and time on each page
%%% comment out if not needed (next 14 lines)
\makeatletter
{\newcount\@hour}
{\newcount\@minute}
\def\timenow{\@hour=\time \divide\@hour by 60
\number\@hour:
  \multiply\@hour by 60 \@minute=\time
  \global\advance\@minute by -\@hour
  \ifnum\@minute<10 0\number\@minute\else
  \number\@minute\fi}
\def\ctimenow{\hfil{\tt \jobname.tex, \today~Time: \timenow }\hfil}
      \let\@oddfoot\ctimenow\let\@evenfoot\ctimenow
\makeatother

\pagestyle{myheadings}
\markboth{{\bf Ideał generowany przez $\aprimeideal$ nie jest równy $\aideal$}}{\bf \today}

% Types message, asks the user to type in a command, then
% defines \answer to be the input instead of executing it.
%%% comment out if not needed (next 13 lines)
%\typein[\answer]{Do you want to include comments? (y/n)}
%
%\newcommand{\annotation}[1]
%  {
%  \if\answer y {{\tt #1}}
%  \fi
%  }
%
%\if\answer y
%\typeout{I shall INCLUDE comments.}
%\else
%\typeout{Comments will be NOT shown.}
%\fi

% To see corrections comment next line and uncomment the second one
\newcommand{\correction}[2]{#1}
% \newcommand{\correction}[2]{#2}

\pagestyle{myheadings}
% To see corrections comment next line and uncomment the second one

\author{Andrzej Nowik}
%\address[]{}

\begin{document}
\title{Ideał generowany przez $\aidealprime$ nie jest równy ideałowi $\aideal$}

%\begin{abstract}
%(...)
%\end{abstract}

\maketitle

\section{Oznaczenia}
$\Phi(E)$ oznacza zbiór punktów gęstości zbioru $E$.
Jeśli $f \colon X \to Y$ jest funkcją to przez
$f[A]$ oznaczamy obraz zbioru $A\subseteq X$, 
zaś $f^{-1}[B]$ przeciwobraz zbioru $B \subseteq Y$.
$\tau_d$ oznacza topologię gęstości (czyli rodzinę zbiorów otwartych w tej topologii).

\section{Ważna Uwaga!!!}
Metoda konstrukcji zbioru $A_1$ jak i dowód 
jego własności jest technicznie kopią metody i pomysłu
użytych w pracy \cite{FG}, Proposition 6.
Tutaj wprowadzono do tej konstrukcji parę
drobnych modyfikacji, jak np. użycie jednej
sekwencji $(\frac{a_n + b_n}{2})$ - w
cytowanej pracy Autorzy używali ,,macierzy ciągów'',
czyli ,,ciągów ciągów'': $(x_n^k)$ oraz $(y_n^k)$),
etc.

\section{Główny wynik}
\begin{theorem}
  Istnieje zbiór przeliczalny $A^* \subseteq \real$ taki że
$A^* \in \aideal$ oraz $A^*$ nie należy do ideału
generowanego przez rodzinę zbiorów $\aidealprime$.
\end{theorem}

\section{Narzędziowe zbiory używane w konstrukcji}

Niech $P \subseteq [0,1]$ będzie zbiorem takim, że:
\begin{itemize}
\item
  $P$ jest zbiorem doskonałym
\item
  $P$ jest zbiorem nigdziegęstym
\item
  $0, 1 \in P$
\item
  miara Lebesgue'a zbioru $P$ jest dodatnia.
\end{itemize}

Ponadto wybierzmy jeden dowolny (ale ustalony) punkt $p^* \in \Phi(P)$
Niech $[0, 1] \setminus P = \bigcup_{n=1}^\infty (a_n, b_n)$ będzie
przedstawieniem zbioru otwartego będącego dopełnieniem zbioru $P$
jako sumę parami rozłącznych przedziałów otwartych.

\section{Pomocnicze lematy}

\begin{lemma}
\label{lemat-o-sumie}
Załóżmy że dla każdego $m = 1,2,\ldots$ mamy ustalony zbiór
$B_m \subseteq (a_m, b_m)$ taki że $B_m \in \aideal$.
Wówczas zbiór $B = \bigcup_{m=1}^{\infty} B_m \in \aideal$.
\end{lemma}

\begin{proof}
Wybierzmy $W \in \tau_d \setminus \lbrace \emptyset \rbrace$.
Bez zmniejszenia ogólności możemy zakładać że 
$W \subseteq (0, 1)$. Załóżmy najpierw że dla pewnego 
$m = 1,2,\ldots$ mamy $(a_m, b_m) \cap W \not= \emptyset$.
Wtedy $(a_m, b_m) \cap W \in \tau_d \setminus \lbrace \emptyset \rbrace$,
więc skoro $B_m \in \aideal$, zatem istnieje zbiór otwarty $U$ 
taki, że $W \cap (a_m, b_m) \cap U \not=\emptyset$
i $B_m \cap W \cap (a_m, b_m) \cap U = \emptyset$. To oznacza że
otwarty zbiór $U_1 = U \cap (a_m, b_m)$ jest taki, że $U_1 \cap W \not=\emptyset$
oraz $U_1 \cap B \cap W = \emptyset$.

Załóżmy zatem że $W \subseteq (0,1) \setminus \bigcup_{n=1}^\infty (a_n, b_n) \subseteq P$.
Wówczas $(0,1) \cap W \not= \emptyset$
natomiast  $(0,1) \cap W \cap B = \emptyset$.
W obu tych przypadkach znaleźliśmy zbiór otwarty $U_1$ taki że
$U_1 \cap W \not=\emptyset$ i $U_1 \cap W \cap B = \emptyset$
co dowodzi że $B \in \aideal$.
\end{proof}

\section{Zasadnicza konstrukcja}

Definiujemy zbiór
\[ A_1 = \lbrace \frac{a_n + b_n}{2} \colon n = 1,2,\ldots\rbrace \cup \lbrace p^* \rbrace.
\]

\flushleft{\it Fakt:} $A_1 \in \aideal \setminus \aidealprime$.
\begin{proof}
Wstawiając w Lemacie \ref{lemat-o-sumie} 
$B_m = \lbrace \frac{a_n + b_n}{2} \rbrace$
i biorąc pod uwagę że singletony należą do $\aideal$ jak i to że
$\aideal$ jest ideałem otrzymujemy natychmiast że $A_1 \in \aideal$.

\smallskip

Przypuśćmy że istnieje zbiór domknięty $E$ taki że
$A_1 \subseteq E \setminus \Phi(E)$.
Wówczas $P \subseteq E$ (gdyż $\forall_{n=1,2\ldots} E \cap (a_n, b_n) \not= \emptyset$).
Więc $p^* \in \Phi(P) \subseteq \Phi(E)$ czyli $A_1 \cap \Phi(E) \not= \emptyset$
co daje sprzeczność.
\end{proof}

Konstruujemy teraz zbiór $A_2$.
Wykorzystywać będziemy zbiór doskonały $P$
użyty uprzednio w konstrukcji zbioru $A_1$.
Dla $n = 1,2,\ldots$ definiujemy odwzorowanie liniowe:
$l_n\colon \real\to\real$ wzorem
\[l_n(x) = \frac{b_n - a_n}{3} \cdot x + \frac{2}{3} a_n + \frac{1}{3} b_n.\]
Widać że odwzorowanie to przeprowadza przedział $[0,1]$ na
środkowy przedział powstały z podziału przedziału $[a_n, b_n]$ na trzy
równe części.

\smallskip

Definiujemy:

\[ A_2 = \bigcup_{m=1}^{\infty} l_m[A_1] \cup \lbrace p^* \rbrace.
\]


\flushleft{\it Fakt 1:} $A_2 \in \aideal$.
\begin{proof}
Wiemy że $l_m[A_1] \in \aideal$, bo zarówno odwzorowanie liniowe $l_m$
jak i $l_m^{-1}$ są homeomorfizmami które zachowują punkty gęstości.
Połóżmy zatem w Lemacie \ref{lemat-o-sumie} 
$B_m = l_m[A_1]$ i z faktu iż $\aideal$ jest ideałem
zawierającym singletony dostajemy natychmiast wniosek
że $A_2 \in \aideal$.

\flushleft{\it Fakt 2:} $A_2$ nie jest sumą dwóch zbiorów z rodziny $\aidealprime$.
Przypuśćmy że $A_2 \subseteq (E_1 \setminus \Phi(E_1)) \cup (E_2 \setminus \Phi(E_2))$
dla pewnych dwóch zbiorów domkniętych $E_1, E_2$.  Przypuśćmy że dla pewnego 
$m = 1,2,\ldots$ mielibyśmy że $l_m[A_1] \cap E_2 = \emptyset$.
Wówczas $l_m[A_1] \subseteq E_1 \setminus \Phi(E_1)$, czyli 
$A_1 \subseteq l_m^{-1}[E_1] \setminus \Phi(l_m^{-1}[E_1])$, gdyż
odwzorowanie liniowe $l_m$ zachowuje punkty gęstości (i tak samo odwzorowanie odwrotne). Ponieważ 
przeprowadza ono też zbiory domknięte na zbiory domknięte (i tak samo odwzorowanie odwrotne)
więc daje to sprzeczość z tym że $A_1 \not\in \aidealprime$.
Zatem $l_m[A_1] \cap E_2 \not= \emptyset$.
Identyczne rozumowanie (ze względu na symetrię założeń o $E_1$ i $E_2$)
pokazuje że także i $l_m[A_1] \cap E_1 \not= \emptyset$.
Ponieważ $l_m[A_1] \subseteq (a_m, b_m)$
więc $P \subseteq E_1$ i $P \subseteq E_2$. 
Mamy zatem $p^* \in \Phi(P) \subseteq \Phi(E_1)$
i $p^* \in \Phi(P) \subseteq \Phi(E_2)$
czyli $p^* \not\in (E_1 \setminus \Phi(E_1)) \cup (E_2 \setminus \Phi(E_2))$
co jest niemożliwe bo $p^* \in A_2$.
\end{proof}

Definiujemy teraz indukcyjne, mając już skonstruowane $A_n$:

\[ A_{n + 1} = \bigcup_{m=1}^{\infty} l_m[A_n] \cup \lbrace p^* \rbrace.
\]

\flushleft{\it Fakt 1:} $A_{n+1} \in \aideal$.
\begin{proof}
Dowód przeprowadzamy przez indukcję. Pierwszy krok indukcyjny 
(że $A_1 \in \aideal$) jak i zresztą przypadek że
$A_2 \in \aideal$ zostały już wykazane wcześnej. 
Zakładamy zatem że $A_n \in \aideal$.

Wiemy że $l_m[A_n] \in \aideal$, bo zarówno odwzorowanie liniowe $l_m$
jak i $l_m^{-1}$ są homeomorfizmami które zachowują punkty gęstości.
Połóżmy zatem w Lemacie \ref{lemat-o-sumie} 
$B_m = l_m[A_n]$ i z faktu iż $\aideal$ jest ideałem
zawierającym singletony dostajemy natychmiast wniosek
że $A_{n+1} \in \aideal$.
\end{proof}

\flushleft{\it Fakt 2:} $A_{n+1}$ nie jest sumą $n + 1$ zbiorów z rodziny $\aidealprime$.
\begin{proof}
Dowód przeprowadzamy indukcyjnie, więc zakładamy że $A_n$ nie jest sumą
$n$ zbiorów z rodziny $\aidealprime$.
Przypuśćmy że $A_{n+1} \subseteq \bigcup_{j=0}^n (E_j \setminus \Phi(E_j)$
dla pewnych dwóch zbiorów domkniętych $E_0, E_1, \ldots, E_n$.  Przypuśćmy że dla pewnego 
$m = 1,2,\ldots$ mielibyśmy że $l_m[A_n] \cap E_0 = \emptyset$.

Wówczas $l_m[A_n] \subseteq \bigcup_{j=1}^n E_j \setminus \Phi(E_j)$, czyli 
$A_n \subseteq \bigcup_{j=1}^n l_m^{-1}[E_j] \setminus \Phi(l_m^{-1}[E_j])$, gdyż
odwzorowanie liniowe $l_m$ zachowuje punkty gęstości (i tak samo odwzorowanie odwrotne). Ponieważ 
przeprowadza ono też zbiory domknięte na zbiory domknięte (i tak samo odwzorowanie odwrotne)
więc daje to sprzeczość z założeniem indukcyjnym że $A_n$ nie jest sumą $n$ zbiorów z rodziny $\aidealprime$.
Zatem $l_m[A_n] \cap E_0 \not= \emptyset$.

Identyczne rozumowanie (ze względu na symetrię założeń o zbiorach $E_j$)
pokazuje że także i $l_m[A_n] \cap E_j \not= \emptyset$
dla każdego $j = 0, \ldots, n$.
Ponieważ $l_m[A_n] \subseteq (a_m, b_m)$
więc $P \subseteq \bigcap_{j=0}^n E_j$. 
Mamy zatem $\forall_{j=0,\ldots, n} p^* \in \Phi(P) \subseteq \Phi(E_j)$
czyli $p^* \not\in \bigcup_{j=0}^n (E_j \setminus \Phi(E_j))$
co jest niemożliwe bo $p^* \in A_{n+1}$.
\end{proof}

Na koniec definiujemy 
\[ A^* = \bigcup_{m=1}^{\infty} l_m[A_m] \cup \lbrace p^* \rbrace.
\]

\flushleft{\it Fakt 1:} $A^* \in \aideal$.
\begin{proof}
Z poprzednich wyników wiemy że $l_m[A_m] \in \aideal$, bo 
$A_m \in \aideal$ a także zarówno odwzorowanie liniowe $l_m$
jak i $l_m^{-1}$ są homeomorfizmami które zachowują punkty gęstości.
Połóżmy zatem w Lemacie \ref{lemat-o-sumie} 
$B_m = l_m[A_m]$ i z faktu iż $\aideal$ jest ideałem
zawierającym singletony dostajemy natychmiast wniosek
że $A^* \in \aideal$.
\end{proof}

\flushleft{\it Fakt 2:} $A^*$ nie jest sumą skończenie
wielu zbiorów z rodziny $\aidealprime$.
\begin{proof}
Przypuśćmy niewprost że
$A^* \subseteq \bigcup_{j=0}^n E_j \setminus \Phi(E_j)$
dla pewnych domkniętych zbiorów $E_j$.
Wówczas mielibyśmy
$l_{n+1}[A_{n+1}] \subseteq \bigcup_{j=0}^n E_j \setminus \Phi(E_j)$,
czyli
$A_{n+1} \subseteq \bigcup_{j=0}^n l_{n+1}^{-1}\big[E_j \setminus \Phi(E_j)\big]$,
ale liniowe odwzorowanie $l_{n+1}$ zachowuje punkty gęstości,
domknięcia zbiorów, zatem mielibyśmy:
$A_{n+1} \subseteq \bigcup_{j=0}^n l_{n+1}^{-1}[E_j] \setminus
\Phi(l_{n+1}^{-1}[E_j])$, co przeczy temu że zbiór
$A_{n+1}$ nie jest sumą $n+1$ zbiorów z rodziny $\aidealprime$.
\end{proof}

%%%%%%%%%%%%%%%%%%%%%%%%%%%%%%%%%%%%%%%%%%%%%%%%%%%%%%%%%%%%%%%%%%%%%%%%%%%%%
%                        The Bibliography                                   %
%%%%%%%%%%%%%%%%%%%%%%%%%%%%%%%%%%%%%%%%%%%%%%%%%%%%%%%%%%%%%%%%%%%%%%%%%%%%%
\begin{thebibliography}{10}

\bibitem[FG]{FG}
M. Frankowska, S.G\l{}\c{a}b, 
\emph{On some $\sigma$-ideal without ccc}, Colloquium Mathematicum \textbf{158 (1)} (2019), 127--140.
\end{thebibliography}

\end{document}
