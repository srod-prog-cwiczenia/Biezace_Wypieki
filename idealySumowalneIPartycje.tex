

\documentclass[12pt]{article}

%---------------------------------------------------------------------
% AMS packages 
%---------------------------------------------------------------------
 
% Packiet loaded automatically by amsart:
%  1. amsmath, 2. amsthm, 3. amsfonts.
 
\usepackage{amssymb}
\usepackage{amsmath}
 
% *** 'amsfonts': 
% boldface of the symbol of the real line: 'R'
\usepackage{amsfonts}
 
% *** 'amsthm': 
% 1. makes easy to modify the macro: \newtheorem{}{} since contains
% \theoremstyle{}
% 2. contains macro: \begin{proof} ... \end{proof}
\usepackage{amsthm} 


%%%%%%%%%%%%%%% Polish letter packages %%%%%%%%%%%%%%%%%%%%%%%%%%%%%%%%
%\usepackage[polish]{babel}
%\usepackage[utf8]{inputenc}
%\usepackage{t1enc}
 
%---------------------------------------------------------------------
% Theorems 
%---------------------------------------------------------------------
 
% Theorem style 'plain' are for: Theorem, Lemma, Corollary, 
% Proposition, Conjecture, Criterion, Algorithm  
\theoremstyle{plain} 
\newtheorem{theorem}{Theorem}[section] 
\newtheorem{lemma}[theorem]{Lemma} 
\newtheorem{corollary}[theorem]{Corollary} 
\newtheorem{conclusion}[theorem]{Corollary} 
\newtheorem{claim}[theorem]{Claim} 
\newtheorem{fact}[theorem]{Fact} 
\newtheorem{proposition}[theorem]{Proposition} 
\newtheorem{axiom}{Axiom} 
 
% Theorem style 'definition' are for: Definition, Condition, Problem, 
% Example 
 
\theoremstyle{definition} 
\newtheorem{definition}[theorem]{Definition} 
\newtheorem{example}[theorem]{Example} 
\newtheorem{exercise}{Exercise} 
\newtheorem*{solution}{Solution} 
 
% Theorem style 'remark' are for: Remark, Note, Notation, Claim, 
% Summary, Acknowledgement, Case, Conclusion 
 
\theoremstyle{remark} 
\newtheorem*{remark}{Remark} 
\newtheorem*{notation}{Notation} 
\newtheorem*{acknowledgement}{Acknowledgement} 

%%%%%%%%%%%%%%%%%  The previous one %%%%%%%%%%%%%%%%%%%%%%%%%%%%%%%%%%%%%%
%\newtheorem{theorem}{Theorem}[section]
%\newtheorem{lemma}[theorem]{Lemma}
%\newtheorem{observation}[theorem]{Observation}
%\newtheorem{conclusion}[theorem]{Conclusion}
%\newtheorem{question}[theorem]{Question}
%\newtheorem{problem}[theorem]{Problem}
%\newtheorem{corollary}[theorem]{Corollary}
%\newtheorem{definition}[theorem]{Definiton}
%\newtheorem{claim}[theorem]{Claim}
%\newtheorem{fact}[theorem]{Fact}
%\newtheorem{conjecture}[theorem]{Conjecture}
%\newtheorem{example}[theorem]{Example}
%\newtheorem{proposition}[theorem]{Proposition}
%\newtheorem{remark}[theorem]{Remark}
%%%%%%%%%%%%%%%%%%%%%%%%%%%%%%%%%%%%%%%%%%%%%%%%%%%%%%%%%%%%%%%%%%%%%


%A
\newcommand{\afc}{AFC}
\newcommand{\afcbar}{\overline{AFC}}
\newcommand{\arr}{\rightarrow}
\newcommand{\Arr}{\Rightarrow}

%B
\newcommand{\baire}{\omega^{\omega}}
\newcommand{\Bor}{\mbox{${\cal B}or$}}
%Previous seems to be much finer than next.
%\newcommand{\Bor}{{\it Bor}}
\newcommand{\borelucrz}{Borel-UCR_0}

%C
\newcommand{\ca}{2^{\omega}}
\newcommand{\cantor}{\ca}
\newcommand{\Card}[1]{\Vert #1 \Vert}

%D
\newcommand{\dom}{{\rm dom}}
\newcommand{\dummy}{{\tt Blah blah blah}}

%E
\newcommand{\Even}{\hbox{\rm \tiny Even}}

%F
\newcommand{\finsub}{[\omega]^{<\omega}}
\newcommand{\forces}{\mathrel{\|}\joinrel\mathrel{-}}

%G
\newcommand{\Graph}{\hbox{\it Graph}}

%H
\newcommand{\homeomorphic}{\approx}

%I
\newcommand{\incr}{\omega^{\uparrow \omega }}
\newcommand{\infsub}{[\omega]^{\omega}}

%L
\newcommand{\la}{\langle}

%M
\newcommand{\meager}{{\cal MGR}}
\newcommand{\minideal}{${\cal F}_{\hbox{\rm \scriptsize min}}(\neg
  D)\;$}

%N
\newcommand{\neglig}{{\cal N}}
\newcommand{\nnatural}{\mathbb{N}}

%O
\newcommand{\Odd}{\hbox{\rm \tiny Odd}}

%P
\newcommand{\Proof}[1]{\bigbreak\noindent{\bf Proof #1}\enspace}

%Q
%%%\newcommand{\qed}{{\hfill\vrule height6pt width6pt depth1pt\medskip}}
%\newcommand{\qed}{\sharp}
\newcommand{\QED}{\hspace{0.1in} \Box \vspace{0.1in}}

%R
\newcommand{\ra}{\rangle}
\newcommand{\ran}{{\rm ran}}
\newcommand{\rational}{\mathbb{Q}}
\newcommand{\real}{\mathbb{R}}

%S
\newcommand{\seq}{\subseteq}
%%%\newcommand{\square}{\hbox{\ \ \ \ \ \vrule\vbox{\hrule\phantom{o}\hrule}\vrule}}
% a small restriction:
\newcommand{\srestriction}{{\hbox{${\scriptstyle\,|\grave{}\,}$}}}

%U
\newcommand{\up}{\uparrow}
\newcommand{\ucrz}{UCR_0}

%%%%%%%%%%%%%%%%%%%%%% Calligraphic font commands %%%%%%%%%%%%%%%%%%%%%%%%%%
\newcommand{\cA}{{\cal A}}
\newcommand{\cB}{{\cal B}}
\newcommand{\cC}{{\cal C}}
\newcommand{\cD}{{\cal D}}
\newcommand{\cE}{{\cal E}}
\newcommand{\cF}{{\cal F}}
\newcommand{\cG}{{\cal G}}
\newcommand{\cH}{{\cal H}}
\newcommand{\cI}{{\cal I}}
\newcommand{\cJ}{{\cal J}}
\newcommand{\cK}{{\cal K}}
\newcommand{\cL}{{\cal L}}
\newcommand{\cM}{{\cal M}}
\newcommand{\cN}{{\cal N}}
\newcommand{\cO}{{\cal O}}
\newcommand{\cP}{{\cal P}}
\newcommand{\cQ}{{\cal Q}}
\newcommand{\cR}{{\cal R}}
\newcommand{\cS}{{\cal S}}
\newcommand{\cT}{{\cal T}}
\newcommand{\cU}{{\cal U}}
\newcommand{\cV}{{\cal V}}
\newcommand{\cW}{{\cal W}}
\newcommand{\cX}{{\cal X}}
\newcommand{\cY}{{\cal Y}}
\newcommand{\cZ}{{\cal Z}}

%%%%%%%%%%%%%%%%%%%%%% Some Greek fonts %%%%%%%%%%%%%%%%%%%%%%%%%%%%%%%%%%%%%%%
\newcommand{\oo}{\omega}
\newcommand{\bb}{\beta}
\newcommand{\dd}{\delta}
\newcommand{\ee}{\varepsilon}
\newcommand{\kk}{\kappa}
\newcommand{\todo}[1]{{\tt #1}}
%%%\newcommand{\th}{\theta}
\newcommand{\Part}{(\omega)^{\leq\omega}}
\newcommand{\FinPart}{(\omega)^{<\omega}}
\newcommand{\InfPart}{(\omega)^{\omega}}

\pagestyle{myheadings}
% To see corrections comment next line and uncomment the second one

\begin{document}
\begin{center}{\bf \Large
		Notes on summable ideals and ideals on partitions
	}
\end{center}
\smallskip
\begin{center}
	by
\end{center}
\smallskip
\begin{center} Pawe\l{} Klinga, Marta Kwela and Andrzej Nowik
\end{center}

\begin{abstract}
	In this paper we prove a selection of theorems on summable ideals as well as ideals on partitions of sets.
	\let\thefootnote\relax\footnote{2010 Mathematics Subject Classification:  Primary: 03E75 Secondary: 03E02.
	}
	\let\thefootnote\relax\footnote{Key words and phrases: ideals, summable ideals, partitions}
\end{abstract}

\section{Summable ideals}

Let us start by introducing necessary definitions and notation. By $\omega$ we will denote the set of natural numbers starting from $0$. An ideal of subsets of $\omega$ is such a family $\cI \subseteq \cP(\omega)$ that
\begin{enumerate}
	\item $\emptyset \in \cI$
	\item $A\in \cI, \: B \in \cI \implies A\cup B \in \cI$
	\item $A\in \cI, \: B\subseteq A \implies B\in \cI$. 
\end{enumerate}
An ideal $\cI$ is proper if $\omega \notin \cI$, i.e. $\cI \neq \cP(\omega)$. Unless stated otherwise, all considered ideals will be proper. From the definition it follows that proper ideals collect \textit{small} subsets of a space, in some sense.

Summable ideals are strictly connected to series of numbers. Namely, for $\sum_{n\in\omega} a_n = \infty, \: a_n>0 \: (n\in\omega)$, let us denote
$$\cI_{(a_n)} = \{ A \subseteq \omega\colon \sum_{n\in A} a_n < \infty\}$$
and call it a summable ideal. A typical example is $\cI_{(\frac{1}{n})}$.

An ideal is tall if for every infinite $A\subseteq\omega$ there is an infinite set $B\subseteq A$ such that $B \in \cI$. It is a well known fact that a summable ideal is tall if and only if its corresponding sequence converges to $0$.
	
For any ideals $\cI, \cJ$ by $\cI \oplus \cJ$ we denote
an ideal consisting of sets $\{A \cup B\colon A \in \cI \wedge B \in \cJ\}$ 
(assuming that this ideal is proper).

First, let us state a simple remark.
\begin{remark}
Suppose that $\cI$ and $\cJ$ are summable ideals, where
$\cI = \cI_{(a_n)}$ and $\cJ = \cI_{(b_n)}$ for some
series $\sum_{n\in\omega} a_n = \sum_{n\in\omega} b_n = \infty$.
Then $\cI \cap \cJ$ is also a summable ideal and
$\cI \cap \cJ = \cI_{(c_n)}$, where $c_n= \max\{a_n, b_n\}$.
\end{remark}

\begin{theorem}
Let $\sum_n \alpha_n = \infty$, $\sum_n \beta_n = \infty$,
$\alpha_n, \beta_n \geq 0$. The following are equivalent:
\begin{enumerate}
\item
  $\cI_{(\alpha_n)} \oplus \cI_{(\beta_n)} \not = P(\omega)$,
\item
  $\sum_n \min\{\alpha_n,\beta_n\} = \infty$.
\end{enumerate}
Also, if those conditions hold, then $\cI_{(\alpha_n)} \oplus \cI_{(\beta_n)} = \cI_{(\gamma_n)}$,
where $\gamma_n = \min\{\alpha_n,\beta_n\}$.
\end{theorem}
\begin{proof}
Let us assume that $\sum_n \min\{\alpha_n,\beta_n\} = \infty$. If 
$\cI_{(\alpha_n)} \oplus \cI_{(\beta_n)} = P(\omega)$, then there would exist a partition of $\omega$ onto $A \cup B$, such that 
$\sum_n \alpha_n < \infty$ and $\sum_n \beta_n < \infty$.
Then $\sum_n \min\{\alpha_n,\beta_n\} \leq \sum_n \alpha_n + \sum_n \beta_n < \infty$,
which contradicts our assumption.
On the other hand, assume that $\sum_n \min\{\alpha_n,\beta_n\} < \infty$. In this case we define $A = \{n \in \omega \colon \alpha_n \leq \beta_n\}$
and $B = \omega \setminus A$. Then 
$A \in \cI_{(\alpha_n)}$ and $B \in \cI_{(\beta_n)}$, hence 
$\cI_{(\alpha_n)} \oplus \cI_{(\beta_n)} = P(\omega)$.
  
Now let us assume that the conditions 1. and 2. hold. Let $X \in \cI_{(\alpha_n)} \oplus \cI_{(\beta_n)}$, then
$X = A \cup B$, where $A \in \cI_{(\alpha_n)}$,
$B \in \cI_{(\beta_n)}$. Therefore, $\sum_{n\in A} \alpha_n < \infty$, 
$\sum_{n\in B} \beta_n < \infty$, so 
$\sum_{n\in X} \gamma_n \leq \sum_{n\in A} \alpha_n + \sum_{n\in B} \beta_n < \infty$.
On the other hand, if $\sum_{n\in X} \gamma_n < \infty$, then we define $A = \{n\in X\colon \alpha_n \leq \beta_n \}$ and 
$B = X \setminus A$. In this case $A \in \cI_{(\alpha_n)}$ and
$B \in \cI_{(\beta_n)}$, hence
$X\in\cI_{(\alpha_n)} \oplus \cI_{(\beta_n)}$.
\end{proof}

\begin{remark}
An analogous version can be formulated for a larger than two (but finite) number of ideals -- for instance, $\cI_{(\alpha_n)} \oplus \cI_{(\beta_n)} \oplus \cI_{(\gamma_n)} = \cI_{(\delta_n)}$, where $\delta_n = \min\{\alpha_n,\beta_n, \gamma_n\}$, etc.
\end{remark}

Let $(\alpha_n^{(i)})$ be a sequence of sequences of nonnegative real numbers. We assume that for every $K \in \omega$ it is true that $\sum_n \min \{\alpha_n^{(1)}, \ldots, \alpha_n^{(K)}\} = \infty$. Then for every $K \in \omega$ we have
$\cI_{(\alpha_n^{(1)})} \oplus\ldots\oplus \cI_{(\alpha_n^{(K)})} \not= P(\omega)$.
In this case we can define an infinite ``amalgam'' of ideals $\cI_{(\alpha_n^{(i)})}$, in the following way:			 
$$\bigoplus_{i}\cI_{(\alpha_n^{(i)})} = \bigcup_{K} \cI_{(\alpha_n^{(1)})} \oplus\ldots\oplus \cI_{(\alpha_n^{(K)})}.$$
The ideal defined in this way is proper, as it is the union of an increasing sequence of proper ideals.\\


It is possible to execute a more general construction. Let us consider a collection $\left\{ \cI_{(\alpha_n^{(t)})} \colon t \in T \right\} $ for a certain set of indices $T$, where $\sum_{n\in\omega}\alpha_n^{(t)} = \infty$ for every $t\in T$. We also assume that for every finite subset $S \subseteq T$ we have $\sum_n \min_{s\in S} \{\alpha_n^{(s)}\} = \infty$.

We choose an increasing sequence of naturals $(N_i)_{i\in\omega}$ according to the following method. We put $N_0 = 0$. Having $(N_j)_{j \leq i}$ chosen, we choose $N_{i + 1} > N_{i}$ so that $\sum_{n = N_{i} + 1}^{N_{i+1}} \min\{\alpha_n^{(0)} , \ldots, \alpha_n^{(i)} \} \geq 1$. We define the series $\sum_n \eta_n$ with: $\eta_n = \min\{\alpha_n^{(0)} , \ldots, \alpha_n^{(i)}\}$, where $i$ is such that $n \in [N_i, N_{i+1})$. We will check that for every $i$ we have $\cI_{(\alpha_n^{(i)})} \subseteq \cI_{(\eta_n)}$. Let $X \subseteq \omega$ be a set such that $\sum_{n\in X} \alpha_n^{(i)} < \infty$. Then for $n\in X, n > N_i$, we have $\alpha_n^{(i)} \geq \min\{\alpha_n^{(0)} , \ldots, \alpha_n^{(i)}\} = \eta_n$. Hence, $\sum_{n\in X}\eta_n < \infty$.\\


Now we give an example to show that $\bigoplus_{i}\cI_{(\alpha_n^{(i)})}$ does not need to be equal to the ideal $\cI_{(\eta_n)}$. We will do this using a matrix notation (where we interpret the matrix as the notation for a sequence of sequences).

\[
  \begin{bmatrix}
     \mathbf{0}  &           0 & 0 & 0 & 0 & 0 & 0 & \ldots\\
    \frac{1}{2} &  0 & 0 & 0 & 0 & 0 & 0 & \ldots\\
    \frac{1}{2} & 0 & 0 & 0 & 0 & 0 & 0 & \ldots\\
    \frac{1}{3} & \mathbf{0} & 0 & 0 & 0 & 0 & 0 & \ldots\\
    \frac{1}{3} & \frac{1}{3} & 0 & 0 & 0 & 0 & 0 & \ldots\\
    \frac{1}{3} & \frac{1}{3} & 0 & 0 & 0 & 0 & 0 & \ldots\\
    \frac{1}{3} & \frac{1}{3} & 0 & 0 & 0 & 0 & 0 & \ldots\\
    \frac{1}{4} & \frac{1}{4} & \mathbf{0} & 0 & 0 & 0 & 0 & \ldots\\
    \frac{1}{4} & \frac{1}{4} & \frac{1}{4} & 0 & 0 & 0 & 0 & \ldots\\
		 \vdots     &      \vdots & \vdots & \vdots & \vdots & \vdots & \vdots & \ddots\\
  \end{bmatrix}
\]

Then $(N_i)_{i\in\omega} = (0, 3, 7, 12, \ldots )$, i.e., $N_i = \frac{i^2 + 5i}{2}$ for every $i\in\omega$.
Let $X= \{N_i \colon i\in\omega\}$. Note that $X \in \cI_{(\eta_n)}$ since $\sum_{n\in X} \eta_n = 0 < \infty$. 
It remains to show that $X\not\in\bigoplus_{i}\cI_{(\alpha_n^{(i)})}$. Observe that all $\cI_{(\alpha_n^{(i)})}$ are equal since the sequences $(\alpha_n^{(i)})$ differ only on finitely many initial terms. Therefore, $\bigoplus_{i}\cI_{(\alpha_n^{(i)})}= \cI_{(\alpha_n^{(0)})}$. However, $X\not\in\cI_{(\alpha_n^{(0)})}$ as $\sum_{n\in X} \alpha_n^{(0)} = 0 +\frac{1}{3} + \frac{1}{4} +\frac{1}{5} +\ldots = \infty$.\\


Now we will give an example to show that $\bigoplus_{i}\cI_{(\alpha_n^{(i)})}$ does not need to be a summable ideal.
	
Let $\{A_i\colon i\in\omega\}$ be a partition of $\omega$ onto sets of cardinality $\aleph_0$. For every $i\in\omega$ let $(a_n^{(i)})_{n\in\omega}$ be an increasing enumaration of elements of the set $A_i$. For every $i\in\omega$ we define the series $\sum_n \alpha_n^{(i)}$ according to the method:
$$\alpha_n^{(i)} = 
\left\{
\begin{array}{ll}
1 & \hbox{\ if\ } x \not\in A_i,\\
\frac{1}{n} & \hbox{\ if\ } x \in A_i \hbox{{\ and\ }} n 
\hbox{{\ is\ (unique)\ such\ that\ }} a_n^{(i)} = x.
\end{array}
\right.$$
Then for every $K \in \omega$ we have: 
$\sum_n \min \{\alpha_n^{(1)}, \ldots, \alpha_n^{(K)}\} = \infty$.
Also, $\bigoplus_{i}\cI_{(\alpha_n^{(i)})}$ is the ideal of sets $X \subseteq \omega$ for which there exists $K$ such that $X \subseteq \bigcup_{i < K} A_i$.
If $i < K$, then $\sum \{\frac{1}{n} \colon a_n^{(i)} \in X\} < \infty$.

We will show that the ideal described above is not summable. Let us denote it by $\cJ$. Let $\sum_n \beta_n = \infty$ be such divergent series of nonnegative reals that $\cJ = \cI_{(\beta_n)}$. For every $i \in \omega$ let us notice that the ideal $\cJ \restriction A_i$ (which is the ideal ``restricted'' to the set $A_i$) is a tall ideal (since the sequence of inverses of respective indices of elements of $A_i$ converges to zero). Therefore, a series $\sum_{m\in A_i} \beta_m $ consists of elements converging to zero (we use a well-known fact that an ideal $\cI_{(\alpha_n)}$ is tall if and only if $\lim_{n\to\infty} \alpha_n = 0$).
If so, then for every $i$ we can pick an element $b_i \in A_i$ such that $\beta_{b_i} \leq \frac{1}{m^2}$. 

We define $B = \lbrace {b_i\colon i\in\omega} \rbrace$. Then $\sum_{m\in B} \beta_m < \infty$, so $B \in \cJ$. Meanwhile, $\forall_{i} |A_i \cap B| = \infty$, which contradicts the definition of the ideal $\cJ$.\\


\section{Ideals on partitions}

Let $\Part$ denote the family of all partitions of the set of natural numbers. By $\FinPart$ we denote the family of partitions of the naturals into a finite number of pieces. The family $\Part$ has natural operations and relations defined in the following way.

For $P,Q \in \Part$ we define:
$P\sqcup Q = \{A \cap B \colon A \in P,\ B\in Q,\ A \cap B \not= \emptyset\}$
and $Q \sqsubseteq P$ when $\forall_{A\in P}\ \exists_{B\in Q}\ A\subseteq B$.

By ${\bf 1}$ we denote the partition of the naturals into singletons. Analogously, by ${\bf 0}$ we denote the partition of the naturals into one element.

\begin{definition}
An ideal in the set of partitions is a non-empty family of partitions $\cP \subseteq \Part$ such that:
\begin{enumerate}
\item ${\bf 1} \not\in\cP$,
\item if $A,B\in \cP$, then $A\sqcup B\in \cP$,
\item if $A \sqsubseteq B$ and $B\in \cP$, then $A\in B$.
\end{enumerate}
\end{definition}
 
Let us mention a number of examples.

The family $\FinPart$ is an ideal (it is a partition counterpart of the classic ideal $Fin$ on $P(\omega)$).
That is why we should analogously require that every partition ideal contains $\FinPart$.

If $\cI\subseteq P(\omega)$ is a (classic) ideal of subsets of the naturals, then it is possible to use it to construct a partition ideal. If $A\subseteq\omega$ is an arbitrary subset of the naturals, then put $\Psi(A) = 
\{[a_k, a_{k+1}) \colon k\in\omega\}$, where
$(a_k)_{k=0}^{\infty}$ is an increasing enumeration of elements of $A$. For any ideal $\cI \subseteq P(\omega)$ define the partition ideal $\Psi(\cI)$ as $\{\Psi(A)\colon A\in\cI\}$.

Furthermore, it is clear that if $A,B\in\cI$, then
$\Psi(A)\sqcup \Psi(B) = \Psi(A \cup B)$
and $A \subseteq B \iff \Psi(A) \sqsubseteq \Psi(B)$.

Keep in mind that the partition ideal $\FinPart$ is \textit{not} of the form $\Psi(\cI)$.\\


A collection of partitions can be treated as a subset of a topological space $2^{\omega\times\omega}$ -- namely, by identifying a given partition with its equivalence relation. Formally, we consider the embedding $r\colon\Part\to 2^{\omega\times\omega}$ defined in the following way:
$$r(A)(\langle m,n\rangle) = 1 \iff \exists_{X\in A}\ x, y \in X.$$

\begin{fact}
The set $\{r(A)\colon A\in\Part\}$ is a closed subset of the space $2^{\omega\times\omega}$.
\end{fact}

\begin{proof}
The conditions in the definition of the equivalence relation are formed as closed sets, for instance, reflexivity as $\{x\in 2^{\omega\times\omega} \colon \forall_{n}\ x(n, n) = 1\}$, symmetry as 
$\bigcap_{n,m\in\omega} \big(\{x\in 2^{\omega\times\omega}\colon 
x(m,n) = 0 \} \cup \{x\in 2^{\omega\times\omega} 
\colon x(n,m) = 1\}\big)$, which are closed sets, just as in the case of transitivity.
\end{proof}

\begin{fact}
The partition ideal $\FinPart$ is an $F_{\sigma}$ subset of $\Part$.
\end{fact}
\begin{proof}
It is sufficient to write
\[\FinPart = \{
x\in 2^{\omega\times\omega} \colon
\exists_{n_1,\ldots,n_k\in\omega}\ 
\forall_{n\in\omega}\ \exists_{i=1,\ldots,k}\ 
x(n_i,n) = 1
\}.\]

If $x \in \omega^\omega$, then we define the partition as
$\{ x^{-1}(k)\colon k\in\omega \} \setminus \{\emptyset\}$.
\end{proof}

Below we show the example of a fairly natural ideal on partitions, which is not of the form $\Psi(\cI)$.

For the set $A \subseteq \omega$ and an arbitrary partition $P\in\Part$ we define a new partition of the set $A$ in the following way:
$$P\restriction A = \{A \cap Z\colon
Z \in P \wedge Z\cap A \not=\emptyset\}.$$
This allows us to define the following partition ideal.
$$\cI\cP = \{P \in \Part\colon \lim_{n\to\infty} \frac{\ln |P \restriction n|}{\ln n} = 0\}.$$
The fact that this family forms an ideal follows from the inequality
$$|(P \sqcup Q) \restriction A| \leq |P \restriction A| \cdot |Q \restriction A|.$$
Then we have 
$$\frac{\ln |(P \sqcup Q) \restriction n|}{\ln n} \leq \frac{\ln |P \restriction n|}{\ln n} + \frac{\ln |Q \restriction n|}{\ln n},$$
so if $P,Q\in\cI\cP$, then also $P \sqcup Q\in\cI\cP$.
Using the logarithm in the denominator of the definition of $\cI\cP$ assures us that the partition ${\bf 1}$ does not belong to $\cI\cP$.
It is easy to see that $\FinPart\subseteq \cI\cP$. 

\newcommand{\nosort}[1]{}
\begin{thebibliography}{1}
	\bibitem{BNFP}
	P. Borodulin-Nadzieja, B. Farkas, G. Plebanek, \newblock Representations of ideals in Polish groups and in Banach spaces, \newblock {\em Journal of Symbolic Logic} {\bf 80}: 1268–1289, 2014.
	
	\bibitem{KK}
	M.I. Kadets, V.M. Kadets, \newblock Series in Banach Spaces: Conditional and Unconditional Convergence, Operator Theory: Advances and Applications. Birkhäuser, Basel, 1997.
	
	\bibitem{KN}
	P. Klinga, A. Nowik,
	\newblock Extendability to summable ideals,
	\newblock {\em Acta Mathematica Hungarica}, {\bf 152}: 150-160, 2017. $\,$
	

\end{thebibliography}


\begin{center}
	\flushleft{{\sl Addresses:}} \\
	Pawe\l{} Klinga \\
	Institute of Mathematics \\
	University of Gda\'nsk \\
	Wita Stwosza 57 \\
	80 -- 952 Gda\'nsk \\
	Poland\\
	e-mail: pawel.klinga@ug.edu.pl
\end{center}

\begin{center}
	\flushleft{{\sl Address:}} \\
	Marta Kwela \\
	Institute of Mathematics \\
	University of Gda\'nsk \\
	Wita Stwosza 57 \\
	80 -- 952 Gda\'nsk \\
	Poland\\
	e-mail: marta.kwela@ug.edu.pl
\end{center}

\begin{center}
	\flushleft{{\sl Address:}} \\
	Andrzej Nowik \\
	Institute of Mathematics \\
	University of Gda\'nsk \\
	Wita Stwosza 57 \\
	80 -- 952 Gda\'nsk \\
	Poland\\
	e-mail: andrzej.nowik@ug.edu.pl
\end{center}


\end{document}  


