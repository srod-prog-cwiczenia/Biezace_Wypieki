

\documentclass[12pt]{article}

%---------------------------------------------------------------------
% AMS packages 
%---------------------------------------------------------------------
 
% Packiet loaded automatically by amsart:
%  1. amsmath, 2. amsthm, 3. amsfonts.
 
\usepackage{amssymb}
\usepackage{amsmath}
 
% *** 'amsfonts': 
% boldface of the symbol of the real line: 'R'
\usepackage{amsfonts}
 
% *** 'amsthm': 
% 1. makes easy to modify the macro: \newtheorem{}{} since contains
% \theoremstyle{}
% 2. contains macro: \begin{proof} ... \end{proof}
\usepackage{amsthm} 


%%%%%%%%%%%%%%% Polish letter packages %%%%%%%%%%%%%%%%%%%%%%%%%%%%%%%%
\usepackage[polish]{babel}
\usepackage[utf8]{inputenc}
\usepackage{t1enc}
 
%---------------------------------------------------------------------
% Theorems 
%---------------------------------------------------------------------
 
% Theorem style 'plain' are for: Theorem, Lemma, Corollary, 
% Proposition, Conjecture, Criterion, Algorithm  
\theoremstyle{plain} 
\newtheorem{theorem}{Theorem}[section] 
\newtheorem{lemma}[theorem]{Lemma} 
\newtheorem{corollary}[theorem]{Corollary} 
\newtheorem{conclusion}[theorem]{Corollary} 
\newtheorem{claim}[theorem]{Claim} 
\newtheorem{fact}[theorem]{Fact} 
\newtheorem{proposition}[theorem]{Proposition} 
\newtheorem{axiom}{Axiom} 
 
% Theorem style 'definition' are for: Definition, Condition, Problem, 
% Example 
 
\theoremstyle{definition} 
\newtheorem{definition}[theorem]{Definition} 
\newtheorem{example}[theorem]{Example} 
\newtheorem{exercise}{Exercise} 
\newtheorem*{solution}{Solution} 
 
% Theorem style 'remark' are for: Remark, Note, Notation, Claim, 
% Summary, Acknowledgement, Case, Conclusion 
 
\theoremstyle{remark} 
\newtheorem*{remark}{Remark} 
\newtheorem*{notation}{Notation} 
\newtheorem*{acknowledgement}{Acknowledgement} 

%%%%%%%%%%%%%%%%%  The previous one %%%%%%%%%%%%%%%%%%%%%%%%%%%%%%%%%%%%%%
%\newtheorem{theorem}{Theorem}[section]
%\newtheorem{lemma}[theorem]{Lemma}
%\newtheorem{observation}[theorem]{Observation}
%\newtheorem{conclusion}[theorem]{Conclusion}
%\newtheorem{question}[theorem]{Question}
%\newtheorem{problem}[theorem]{Problem}
%\newtheorem{corollary}[theorem]{Corollary}
%\newtheorem{definition}[theorem]{Definiton}
%\newtheorem{claim}[theorem]{Claim}
%\newtheorem{fact}[theorem]{Fact}
%\newtheorem{conjecture}[theorem]{Conjecture}
%\newtheorem{example}[theorem]{Example}
%\newtheorem{proposition}[theorem]{Proposition}
%\newtheorem{remark}[theorem]{Remark}
%%%%%%%%%%%%%%%%%%%%%%%%%%%%%%%%%%%%%%%%%%%%%%%%%%%%%%%%%%%%%%%%%%%%%


%A
\newcommand{\afc}{AFC}
\newcommand{\afcbar}{\overline{AFC}}
\newcommand{\arr}{\rightarrow}
\newcommand{\Arr}{\Rightarrow}

%B
\newcommand{\baire}{\omega^{\omega}}
\newcommand{\Bor}{\mbox{${\cal B}or$}}
%Previous seems to be much finer than next.
%\newcommand{\Bor}{{\it Bor}}
\newcommand{\borelucrz}{Borel-UCR_0}

%C
\newcommand{\ca}{2^{\omega}}
\newcommand{\cantor}{\ca}
\newcommand{\Card}[1]{\Vert #1 \Vert}

%D
\newcommand{\dom}{{\rm dom}}
\newcommand{\dummy}{{\tt Blah blah blah}}

%E
\newcommand{\Even}{\hbox{\rm \tiny Even}}

%F
\newcommand{\finsub}{[\omega]^{<\omega}}
\newcommand{\forces}{\mathrel{\|}\joinrel\mathrel{-}}

%G
\newcommand{\Graph}{\hbox{\it Graph}}

%H
\newcommand{\homeomorphic}{\approx}

%I
\newcommand{\incr}{\omega^{\uparrow \omega }}
\newcommand{\infsub}{[\omega]^{\omega}}

%L
\newcommand{\la}{\langle}

%M
\newcommand{\meager}{{\cal MGR}}
\newcommand{\minideal}{${\cal F}_{\hbox{\rm \scriptsize min}}(\neg
  D)\;$}

%N
\newcommand{\neglig}{{\cal N}}
\newcommand{\nnatural}{\mathbb{N}}

%O
\newcommand{\Odd}{\hbox{\rm \tiny Odd}}

%P
\newcommand{\Proof}[1]{\bigbreak\noindent{\bf Proof #1}\enspace}

%Q
%%%\newcommand{\qed}{{\hfill\vrule height6pt width6pt depth1pt\medskip}}
%\newcommand{\qed}{\sharp}
\newcommand{\QED}{\hspace{0.1in} \Box \vspace{0.1in}}

%R
\newcommand{\ra}{\rangle}
\newcommand{\ran}{{\rm ran}}
\newcommand{\rational}{\mathbb{Q}}
\newcommand{\real}{\mathbb{R}}

%S
\newcommand{\seq}{\subseteq}
%%%\newcommand{\square}{\hbox{\ \ \ \ \ \vrule\vbox{\hrule\phantom{o}\hrule}\vrule}}
% a small restriction:
\newcommand{\srestriction}{{\hbox{${\scriptstyle\,|\grave{}\,}$}}}

%U
\newcommand{\up}{\uparrow}
\newcommand{\ucrz}{UCR_0}

%%%%%%%%%%%%%%%%%%%%%% Calligraphic font commands %%%%%%%%%%%%%%%%%%%%%%%%%%
\newcommand{\cA}{{\cal A}}
\newcommand{\cB}{{\cal B}}
\newcommand{\cC}{{\cal C}}
\newcommand{\cD}{{\cal D}}
\newcommand{\cE}{{\cal E}}
\newcommand{\cF}{{\cal F}}
\newcommand{\cG}{{\cal G}}
\newcommand{\cH}{{\cal H}}
\newcommand{\cI}{{\cal I}}
\newcommand{\cJ}{{\cal J}}
\newcommand{\cK}{{\cal K}}
\newcommand{\cL}{{\cal L}}
\newcommand{\cM}{{\cal M}}
\newcommand{\cN}{{\cal N}}
\newcommand{\cO}{{\cal O}}
\newcommand{\cP}{{\cal P}}
\newcommand{\cQ}{{\cal Q}}
\newcommand{\cR}{{\cal R}}
\newcommand{\cS}{{\cal S}}
\newcommand{\cT}{{\cal T}}
\newcommand{\cU}{{\cal U}}
\newcommand{\cV}{{\cal V}}
\newcommand{\cW}{{\cal W}}
\newcommand{\cX}{{\cal X}}
\newcommand{\cY}{{\cal Y}}
\newcommand{\cZ}{{\cal Z}}

%%%%%%%%%%%%%%%%%%%%%% Some Greek fonts %%%%%%%%%%%%%%%%%%%%%%%%%%%%%%%%%%%%%%%
\newcommand{\oo}{\omega}
\newcommand{\bb}{\beta}
\newcommand{\dd}{\delta}
\newcommand{\ee}{\varepsilon}
\newcommand{\kk}{\kappa}
\newcommand{\todo}[1]{{\tt #1}}
%%%\newcommand{\th}{\theta}
\newcommand{\Part}{(\omega)^{\leq\omega}}
\newcommand{\FinPart}{(\omega)^{<\omega}}
\newcommand{\InfPart}{(\omega)^{\omega}}

\pagestyle{myheadings}
% To see corrections comment next line and uncomment the second one

\begin{document}
Let us denote
  $\cI_{(a_n)} = \{ A \subseteq \nnatural\colon \sum_{n\in A} a_n < \infty\}$.
For any ideals $\cI, \cJ$ by $\cI \oplus \cJ$ we denote
an ideal $\{A \cup B\colon A \in \cI \wedge B \in \cJ\}$ 
(assuming that this ideal is proper).

At first, let us formulate a simly remark/observation:
\begin{remark}
Suppose that $\cI$ and $\cJ$ are summable ideals, where
$\cI = \cI_{(a_n)}$ and $\cJ = \cI_{(b_n)}$ for some
series $\sum_{n\in\omega} a_n = \sum_{n\in\omega} b_n = \infty$.
Then $\cI \cap \cJ$ is also a summable ideal and
$\cI \cap \cJ = \sum_n \max\{\alpha_n,\beta_n\}$.
\end{remark}

\begin{theorem}
Niech $\sum_n \alpha_n = \infty$, $\sum_n \beta_n = \infty$,
$\alpha_n, \beta_n \geq 0$. N.w.s.r:
\begin{enumerate}
\item
  $\cI_{(\alpha_n)} \oplus \cI_{(\beta_n)} \not = P(\nnatural)$;
\item
  $\sum_n \min\{\alpha_n,\beta_n\} = \infty$
\end{enumerate}
Ponadto jeśli zachodzi warunek z podanej powyżej serii równoważnych warunków
to $\cI_{(\alpha_n)} \oplus \cI_{(\beta_n)} = \cI_{(\gamma_n)}$,
gdzie $\gamma_n = \min\{\alpha_n,\beta_n\}$.
\end{theorem}
\begin{proof}
Załóżmy że $\sum_n \gamma_n = \infty$, gdyby 
$\cI_{(\alpha_n)} \oplus \cI_{(\beta_n)} = P(\nnatural)$ to wtedy
istniałaby partycja $\nnatural$ na zbiory $A \cup B$ taka, że 
$\sum_n \alpha_n < \infty$ oraz $\sum_n \beta_n < \infty$.
Wówczas $\sum_n \gamma_n \leq \sum_n \alpha_n + \sum_n \beta_n < \infty$,
czyli sprzeczność z założeniem. 
Na odwrót, załóżmy że $\sum_n \min\{\alpha_n,\beta\} < \infty$,
definiujemy wówczas $A = \{n \in \nnatural \colon \alpha_n \leq \beta_n\}$
i $B = \nnatural \setminus A$. Wówczas 
$A \in \cI_{(\alpha_n)}$ oraz $B \in \cI_{(\beta_n)}$, czyli 
$\cI_{(\alpha_n)} \oplus \cI_{(\beta_n)} = P(\nnatural)$.
  
Załóżmy że zachodza warunki 1. oraz 2. z pierwszej części 
sformułowania tego twierdzenia. Niech 
$X \in \cI_{(\alpha_n)} \oplus \cI_{(\beta_n)}$, zatem
$X = A \cup B$, gdzie $A \in \cI_{(\alpha_n)}$,
$B \in \cI_{(\beta_n)}$. Stąd $\sum_{n\in A} \alpha_n < \infty$, 
$\sum_{n\in B} \beta_n < \infty$, więc 
$\sum_{n\in X} \gamma_n \leq \sum_{n\in A} \alpha_n + \sum_{n\in B} \beta_n < \infty$.
Na odwrót, jeśli $\sum_{n\in X} \gamma_n < \infty$, to 
definiujemy $A = \{n\in X\colon \alpha \leq \beta \}$,
$B = X \setminus A$. Wówczas $A \in \cI_{(\alpha_n)}$ i
$B \in \cI_{(\beta_n)}$, więc 
$X\in\cI_{(\alpha_n)} \oplus \cI_{(\beta_n)}$.
\end{proof}

Uwaga: Analogicznie można sformułować wersję tego dla większej
niż dwa (ale skończonej) liczby ideałów, np. 
$\cI_{(\alpha_n)} \oplus \cI_{(\beta_n)} \oplus \cI_{(\gamma_n)} = \cI_{(\delta_n)}$,
gdzie $\delta_n = \min\{\alpha_n,\beta_n, \gamma_n\}$, etc.

Niech $(\alpha_n^{(k)})$ będzie podwójnym ciągiem nieujemnych liczb
rzeczywistych. Zakładamy że zachodzi warunek
iż dla każdego $K \in \nnatural$, 
$\sum_n \min \{\alpha_n^{(1)}, \ldots, \alpha_n^{(K)}\} = \infty$.
Wówczas dla każdego $K \in \nnatural$ mamy
$\cI_{(\alpha_n^{(1)})} \oplus,\ldots,\oplus \cI_{(\alpha_n^{(K)})} \not= P(\nnatural)$.
Możemy w takim przypadku zdefiniować ,,nieskończony zlepek'' ideałów 
$\cI_{(\alpha_n^{(k)})}$, przepisem:			 
	
$\bigoplus_{i}\cI_{(\alpha_n^{(i)})} = \bigcup_{K} \cI_{(\alpha_n^{(1)})} \oplus,\ldots,\oplus \cI_{(\alpha_n^{(K)})}$.
Tak zdefiniowany ideał jest właściwy bo jest sumą wstępującego ciągu ideałów właściwych.

Możliwa jest bardziej ogólna 
definicja (konstrukcja): 
Załóżmy że 
$\langle \cI_{(\alpha_n^{(t)})}
\colon t \in T
$ dla pewnego zbioru indeksów
$T$, gdzie 
$\sum_{n\in\omega}\alpha_n^{(t)}) = \infty$ dla każdego $t\in T$.
Zakładamy ponadto że 
dla każdego skończonego podzbioru
$S \subseteq T$ mamy
$\sum_n \min_{s\in S}
\{\alpha_n^{(s)}\} = \infty$.

Wybieramy rosnący ciąg liczb naturalnych $(N_i)_{i\in\nnatural}$ według następującego przepisu: kładziemy $N_0 = 0$. Mając już wybrane $(N_j)_{j \leq i}$
wybieramy $N_{i + 1} > N_{i}$ tak aby $\sum_{n = N_{i} + 1}^{N_{i+1}} \min\{\alpha_n^{(0)} , \ldots, \alpha_n^{(i)} \} \geq 1$.
Definiujemy szereg: $\eta_n = \min\{\alpha_n^{(0)} , \ldots, \alpha_n^{(i)}\}$, gdzie $i$ jest
takie że $n \in [N_i, N_{i+1})$. Sprawdzimy że dla każdego $i$ mamy 
$\cI_{(\alpha_n^{(i)})} \subseteq \cI_{(\eta_n)}$.				
Niech więc $X \subseteq \nnatural$ będzie zbiorem takim że 
$\sum_{n\in X} \alpha_n^{(i)} < \infty$. Wówczas dla $n\in X, n > N_i$ mamy
$\alpha_n^{(i)} \geq \min\{\alpha_n^{(0)} , \ldots, \alpha_n^{(i)}\} = \eta_n$.
Stąd $\sum_{n\in X}\eta_n < \infty$.

Przykład że $\bigoplus_{i}\cI_{(\alpha_n^{(i)})}$ nie musi być
równe ideałowi $\cI_{(\eta_n)}$: 
Odpowiedni przykład podamy używając zapisu macierzowego
(przy czym macierz interpretujemy jako zapis ciągu szeregów).
	
%% tutaj przydałby się jakiś formalny opis jak definujemy narysowaną poniżej macierz szeregów

\[
  \begin{bmatrix}
         1      &           0 & 0 & 0 & 0 & 0\\
    \frac{1}{2} &  \mathbf{0} & 0 & 0 & 0 & 0\\
    \frac{1}{2} & \frac{1}{2} & 0 & 0 & 0 & 0\\
    \frac{1}{2} & \frac{1}{2} & 0 & 0 & 0 & 0\\
    \frac{1}{3} & \frac{1}{3} & \mathbf{0} & 0 & 0 & 0\\
    \frac{1}{3} & \frac{1}{3} & \frac{1}{3} & 0 & 0 & 0\\
    \frac{1}{3} & \frac{1}{3} & \frac{1}{3} & 0 & 0 & 0\\
    \frac{1}{3} & \frac{1}{3} & \frac{1}{3} & 0 & 0 & 0\\
    \frac{1}{4} & \frac{1}{4} & \frac{1}{4} & \mathbf{0} & 0 & 0\\
  \end{bmatrix}
\]
Wówczas $N_i = \{0, 3, 7, 12, \ldots \}$ \todo{Czy to tak faktycznie jest?}.
Niech $X = \{0, 3, 7, 12, \ldots\} = \{ \frac{4 \cdot n^2 + 5 \cdot n}{3} \colon n \in\nnatural \}$.
Wówczas $X \in \cI_{(\eta_n)}$, zaś $X\not\in\cI_{(\alpha_n^{(i)})}$

Przykład na to że $\bigoplus_{i}\cI_{(\alpha_n^{(i)})}$ nie musi być
ideałem sumowalnym:
	
Niech $\{A_i\colon i\in\nnatural\}$ będzie partycją $\nnatural$
na zbiory mocy $\aleph_0$. Dla każdego $i\in\nnatural$ niech
$(a_n^{(i)})_{n\in\nnatural}$ będzie rosnącą numeracją
elementów zbioru $A_i$.
Dla każdego $i\in\nnatural$
definiujemy szereg $\sum_n \alpha_n^{(i)}$ według przepisu:
$\alpha_n^{(i)} = 
\left\{
\begin{array}{cc}
1 & \hbox{\ gdy\ } x \not\in A_i\\
\frac{1}{n} & \hbox{\ gdy\ } x \in A_i \hbox{{\ a\ }} n 
\hbox{{\ to\ (jedyne)\ takie\ że\ }} a_n^{(i)} = x.
\end{array}
\right.$

Wówczas dla każdego $K \in \nnatural$ mamy: 
$\sum_n \min \{\alpha_n^{(1)}, \ldots, \alpha_n^{(K)}\} = \infty$.
Ponadto 
$\bigoplus_{i}\cI_{(\alpha_n^{(i)})}$ to ideał zbiorów
$X \subseteq \omega$ dla którego istnieje $K$
takie że 
$X \subseteq \bigcup_{i < K} A_i$.
oraz gdy $i < K$ to 
$\sum \{\frac{1}{n} \colon a_n^{(i)} \in X\} < \infty$


Sprawdzimy że opisany powyżej ideał nie jest ideałem 
sumowalnym. Oznaczmy go przez $\cJ$. Niech więc
$\sum \beta_n = \infty$ będzie takim rozbieżnym szeregiem
liczb nieujemnych, że $\cJ = \cI_{(\beta_n)}$.
Dla każdego $i \in \omega$ zauważmy że ideał 
$\cJ \restriction A_i$ (czyli ideał ,,obcięty'' do zbioru $A_i$)
jest ideałem ''tall'' (bo ciąg odwrotności kolejnych numerów
elementów zbioru $A_i$ zbiega do zera). 
Zatem odpowiedni szereg $\sum_{m\in A_i} \beta_m $
składa się z elementów dążących do zera 
(korzystamy tu ze znanego faktu iż ideał
$\cI_{(\alpha_n)}$ jest ''tall'' wtedy i tylko wtedy
gdy $\lim_{n\to\infty} \alpha_n = 0$).
Skoro tak to dla każdego $i$ możemy wybrać element
zbiór $b_i \in A_i$ taki, że 
$\beta_{b_i} \geq \frac{1}{m^2}$.
Definiujemy $B = \lbrace {b_i\colon i\in\omega} \rbrace$. Wówczas
$\sum_{m\in B} \beta_m < \infty$, zatem $B \in \cJ$.
Tymczasem $\forall_{i} |A_i \cap B| = \infty$,
więc jest to sprzeczne z definicją ideału $\cJ$.


Możemy zdefiniować relację równoważności oraz relację porządku w rodzinie
rozbieżnych szeregów według przepisu:
$\sum_{(\alpha_n)} \simeq \sum_{(\beta_n)}$ wtedy i tylko wtedy gdy
$\cI_{(\alpha_n)} = \cI_{(\beta_n)}$, oraz
$\sum_{(\alpha_n)} \lhd \sum_{(\beta_n)}$ wtedy i tylko wtedy gdy
$\cI_{(\alpha_n)} \subseteq \cI_{(\beta_n)}$.


\section{Ideały na partycjach}

Niech $\Part$ oznacza rodzinę wszystkich partycji zbioru liczb naturalnych.
Przez $\FinPart$ oznaczamy rodzinę wszystkich partycji
zbioru liczb naturalnych na skończoną liczbę części.
Rodzina $\Part$ posiada naturalne działania oraz relacje
zdefiniowane następująco:

Jeśli $P,Q \in Part$ to definiujemy:
$P\sqcup Q = \{A \cap B \colon A \in P, B\in Q \wedge 
A \cap B \not= \emptyset$
oraz $Q \sqsubseteq P$ gdy $\forall_{A\in P} \exists_{B\in Q} A\subseteq B$.

Przez ${\bf 1}$ oznaczamy partycję zbioru liczb naturalnych 
na singletony. Analogicznie, przez ${\bf 0}$ oznaczamy
partycję zbioru liczb naturalnych na jeden element.

\begin{definition}
{\bf Ideałem} w zbiorze partycji nazywamy niepustą rodzinę
partycji $\cP \subseteq \Part$ taką że
\begin{enumerate}
\item ${\bf 1} \not\in\cP$
\item jeśli $A,B\in \cP$ to $A\sqcup B\in \cP$
\item jeśli $A \sqsubseteq B$ i $B\in \cP$ to $A\in B$.
\end{enumerate}
\end{definition}

Przykłady:

Rodzina $\FinPart$ jest ideałem partycyjnym (analog ideału $Fin$).

Jeśli $\cI\subseteq P(\omega)$ jest (klasycznym) ideałem 
podzbiorów liczb naturalnych to ideał zdefiniowany 
według przepisu: 
Jeśli $A\subseteq\omega$ jest dowolnym podzbiorem
zbioru liczb naturalnych
to definiujemy $\Psi(A) = 
\{[a_k, a_{k+1}) \colon k\in\omega\}$, gdzie
$(a_k)_{k=0}^{\infty}$ jest rosnącą numeracją
elementów zbioru $A$.

Dla dowolnego ideału $\cI \subseteq P(\omega)$
definiujemy ideał partycyjny 
$\Psi(\cI)$ jako $\Psi(A)\colon A\in\cI$.

Co więcej, widać że jeśli $A,B\in\cI$ to
$\Psi(A)\sqcup \Psi(B) = \Psi(A \cup B)$
oraz że $A \subseteq B \iff \Psi(A) \sqsubseteq \Psi(B)$.

Uwaga: ideał partycyjny $\FinPart$ nie jest postaci
$\Psi(\cI)$.

Zbiór partycji można traktować jako podzbiór przestrzeni
topologicznej $2^{\omega\times\omega}$, mianowicie
utożsamiając daną partycję z jej relacją równoważności,
czyli formalnie mamy następujące zanurzenie:
$r\colon\Part\to 2^{\omega\times\omega}$
zdefiniowane według przepisu:

$r(A)(\langle m,n\rangle) = 1 \iff \exists_{X\in A} x, y \in X$.

Mamy:
\begin{fact}
Zbiór $\{r(A)\colon A\in\Part\}$ jest domkniętymy podzbiorem 
przestrzeni $2^{\omega\times\omega}$.
\end{fact}
\proof
Warunki na bycie relacją równoważności wyrażają się
jako zbiory domknięte, np:
zwrotność jako $\{x\in 2^{\omega\times\omega} \forall_{n} x(\langle n, n
\rangle = 1\}$, symetria jako 
$\bigcap_{n,m\in\omega} \big(\{x\in 2^{\omega\times\omega}\colon 
x(m,n) = 0 \} \cup \{x\in 2^{\omega\times\omega} 
\colon x(n,m) = 1\}\big)$, więc zbiór domknięty, tak
samo w przypadku przechodniości relacji.

Mamy:
\begin{fact}
Ideał partycyjny $\FinPart$ jest podzbiorem
typu $F_{\sigma}$ w zbiorze $\Part$.
\end{fact}
\proof
Wystarczy zapisać:
\[\FinPart = \{
x\in 2^{\omega\times\omega}
\exists_{n_1,\ldots,n_k\in\omega}\colon
\forall_{n\in\omega} \exists_{i=1,\ldots,k}
x(n_i,n) = 1
\}\]

Jeśli $x \in \omega^\omega$ to wówczas definiujemy
partycję postaci 
$\{ x^{-1}(k)\colon k\in\omega \} \setminus \{\emptyset\}$.
i oznaczamy ją przez $\rho(x)$.

Poniżej przykład w miarę naturalnego ideału na 
partycjach który nie jest postaci $\Psi(\cI)$.

Dla zbioru $A \subseteq \omega$ i dla dowolnej
partycji $P\in\Part$ definiujemy
partycję zbioru $A$ według przepisu:
$P\restriction A = \{A \cap Z\colon
Z \in P \wedge Z\cap A \not=\emptyset\}$.
Definiujemy teraz następujący ideał
partycyjny:

$\cI\cP = \{P \in \Part\colon \lim \frac{\ln |P \restriction n|}{\ln n} = 0\}$.
Fakt że ta rodzina jest ideałem wynika z nierówności
$|(P \sqcup Q) \restriction A| \leq |P \restriction A| \cdot |Q \restriction A|$.
Mamy wówczas 
$\frac{\ln |(P \sqcup Q) \restriction n|}{\ln n} \leq $
$\frac{\ln |P \restriction n|}{\ln n} + \frac{\ln |Q \restriction n|}{\ln n}$,
więc gdy $P,Q\in\cI\cP$ to wtedy i $P \sqcup Q\in\cI\cP$.
Użycie w definicji ideału $\cI\cP$ logarytmu 
w mianowniku zapewnia że partycja ${\bf 1}$ nie
należy do $\cI\cP$.
Łatwo zauważyć że $\FinPart\subseteq \cI\cP$. 

\end{document}  


