%%%%%%%%%%%%%%%%%%%%%%%%%%%%%%%%%%%%%%%%%%%%%%%%%%%%%%%%%%%%%%%%%%%%%%%%%
%                                                                       %
%                  praca o ideale (a)                                   %
%                                                                       %
%%%%%%%%%%%%%%%%%%%%%%%%%%%%%%%%%%%%%%%%%%%%%%%%%%%%%%%%%%%%%%%%%%%%%%%%%
%\documentclass[12pt]{article}
%%% aby uzyc amsart w pelnej krasie:
% 1. odkomentowac \address i \email
% 2. odkomentowac \keywords{} \subjclass
\documentclass[12pt]{amsart}
\usepackage{amssymb}
\usepackage{amsmath}
\usepackage{latexsym}
\usepackage{amsfonts}
\usepackage{enumerate}
\usepackage{mathtools}
%\usepackage[mathscr]{eucal}
\usepackage[mathscr]{euscript}
\usepackage{eqlist}
\usepackage{amsthm}
%%%%%%%%%%%%%%% Polish letter packages %%%%%%%%%%%%%%%%%%%%%%%%%%%%%%%%
\usepackage[polish]{babel}
\usepackage[utf8]{inputenc}
\usepackage{t1enc}
%%% pakiet do generowania belkotu (wypelniacza):
\usepackage{lipsum}
%%% pakiet do dodawania fikusnych notek `a la ``todo''
\usepackage{todonotes}
%---------------------------------------------------------------------
% Theorems 
%---------------------------------------------------------------------
 
% Theorem style 'plain' are for: Theorem, Lemma, Corollary,
% Proposition, Conjecture, Criterion, Algorithm
\theoremstyle{plain}
\newtheorem{theorem}{Theorem}[section]
\newtheorem{lemma}[theorem]{Lemat}
\newtheorem{corollary}[theorem]{Corollary}
\newtheorem{conclusion}[theorem]{Corollary}
\newtheorem{claim}[theorem]{Claim}
\newtheorem{fact}[theorem]{Fact}
\newtheorem{proposition}[theorem]{Proposition}
\newtheorem{axiom}{Axiom}
% Theorem style 'definition' are for: Definition, Condition, Problem,
% Example
\theoremstyle{definition}
\newtheorem{definition}[theorem]{Definition}
\newtheorem{example}[theorem]{Example}
\newtheorem{exercise}{Exercise}
\newtheorem*{solution}{Solution}
\newtheorem{remark}[theorem]{Remark}
\newtheorem{Problem}[theorem]{Problem}
% Theorem style 'remark' are for: Remark, Note, Notation, Claim,
% Summary, Acknowledgement, Case, Conclusion
\theoremstyle{remark}
\newtheorem*{notation}{Notation}
\newtheorem*{acknowledgement}{Acknowledgement}

%%%%%% makra:

%A
\newcommand{\afc}{AFC}
\newcommand{\afcbar}{\overline{AFC}}
\newcommand{\arr}{\rightarrow}
\newcommand{\Arr}{\Rightarrow}

%B
\newcommand{\baire}{\omega^{\omega}}
\newcommand{\Baire}{\mathfrak{Baire}}
\newcommand{\Bor}{\mbox{${\mathcal B}or$}}
%Previous seems to be much finer than next.
%\newcommand{\Bor}{{\it Bor}}
\newcommand{\borelucrz}{Borel-UCR_0}

%C
\newcommand{\ca}{2^{\omega}}
\newcommand{\cantor}{\ca}
\newcommand{\Card}[1]{\Vert #1 \Vert}
\newcommand{\cl}{\mathit{cl}}

%D
\newcommand{\dom}{{\rm dom}}
\newcommand{\dummy}{{\tt Blah blah blah}}

%E
\newcommand{\Even}{\hbox{\rm \tiny Even}}

%F
\newcommand{\finsub}{[\omega]^{<\omega}}
\newcommand{\forces}{\mathrel{\|}\joinrel\mathrel{-}}

%G
\newcommand{\Graph}{\hbox{\it Graph}}

%H
\newcommand{\homeomorphic}{\approx}

%I
\newcommand{\incr}{\omega^{\uparrow \omega }}
\newcommand{\infsub}{[\omega]^{\omega}}

%L
\newcommand{\la}{\langle}

%M
\newcommand{\meager}{{\mathcal{MGR}}}
\newcommand{\minideal}{${\cal F}_{\hbox{\rm \scriptsize min}}(\neg
  D)\;$}

%N
\newcommand{\neglig}{{\cal N}}
\newcommand{\nnatural}{\mathbb{N}}

%O
\newcommand{\Odd}{\hbox{\rm \tiny Odd}}

%P
\newcommand{\Part}{{\it Part}}
\newcommand{\Perf}{{\it Perf}}
%%%\newcommand{\proof}{\flushleft{ \sc Proof. } \\ }
\newcommand{\Proof}[1]{\bigbreak\noindent{\bf Proof #1}\enspace}

%Q
%%%\newcommand{\qed}{{\hfill\vrule height6pt width6pt depth1pt\medskip}}
%\newcommand{\qed}{\sharp}
\newcommand{\QED}{\hspace{0.1in} \Box \vspace{0.1in}}

%R
\newcommand{\ra}{\rangle}
\newcommand{\ran}{{\rm ran}}
\newcommand{\rational}{\mathbb{Q}}
\newcommand{\real}{\mathbb{R}}

%S
\newcommand{\seq}{\subseteq}
%%%\newcommand{\square}{\hbox{\ \ \ \ \ \vrule\vbox{\hrule\phantom{o}\hrule}\vrule}}
% a small restriction:
\newcommand{\srestriction}{{\hbox{${\scriptstyle\,|\grave{}\,}$}}}
\newcommand{\supp}{\mathit{supp}}

%U
\newcommand{\up}{\uparrow}
\newcommand{\ucrz}{UCR_0}

%%%%%%%%%%%%%%%%%%%%%% Calligraphic font commands %%%%%%%%%%%%%%%%%%%%%%%%%%
\newcommand{\cA}{{\mathcal A}}
\newcommand{\cB}{{\mathcal B}}
\newcommand{\cC}{{\mathcal C}}
\newcommand{\cD}{{\mathcal D}}
\newcommand{\cE}{{\mathcal E}}
\newcommand{\cF}{{\mathcal F}}
\newcommand{\cG}{{\mathcal G}}
\newcommand{\cH}{{\mathcal H}}
\newcommand{\cI}{{\mathcal I}}
\newcommand{\cJ}{{\mathcal J}}
\newcommand{\cK}{{\mathcal K}}
\newcommand{\cL}{{\mathcal L}}
\newcommand{\cM}{{\mathcal M}}
\newcommand{\cN}{{\mathcal N}}
\newcommand{\cO}{{\mathcal O}}
\newcommand{\cP}{{\mathcal P}}
\newcommand{\cQ}{{\mathcal Q}}
\newcommand{\cR}{{\mathcal R}}
\newcommand{\cS}{{\mathcal S}}
\newcommand{\cT}{{\mathcal T}}
\newcommand{\cU}{{\mathcal U}}
\newcommand{\cV}{{\mathcal V}}
\newcommand{\cW}{{\mathcal W}}
\newcommand{\cX}{{\mathcal X}}
\newcommand{\cY}{{\mathcal Y}}
\newcommand{\cZ}{{\mathcal Z}}
%%%%%%%%%%%%%%%%%%%%%% inne %%%%%%%%%%%%%%%%%%%%%%%%%%%%%%%%%%%%%%%
\newcommand{\SqrFr}{\mathbb{SF}}
\newcommand{\Primes}{\mathit{Primes}}
\newcommand{\mathint}{\mathit{int}}
\newcommand{\mathcl}{\mathit{cl}}
\newcommand{\exampleGFnotK}{\mathit{Ex}}
%%%%%%%%%%%%%%%%%%%%%% Some Greek fonts %%%%%%%%%%%%%%%%%%%%%%%%%%%%%%%%%%%%%%%
\newcommand{\oo}{\omega}
\newcommand{\bb}{\beta}
\newcommand{\dd}{\delta}
\newcommand{\ee}{\varepsilon}
\newcommand{\kk}{\kappa}
%%%\newcommand{\th}{\theta}
%%%% makra specyficzne dla tej pracy:
\newcommand{\aideal}{\mathit{(a)}}
\newcommand{\aidealprime}{\mathit{(a^\prime)}}
\newcommand{\Afield}{\mathit{(A)}}
\newcommand{\topWithoutEmptyset}[1]{#1\setminus\lbrace\emptyset\rbrace}
\newcommand{\ND}{\mathsf{ND}}
\newcommand{\biMB}{bi-Marczewski-Burstin}
\newcommand{\tauEllentuck}{\tau_{\mathrm{EL}}}
\newcommand{\baseEllentuck}{\cB_{\mathrm{EL}}}
\newcommand{\ninomega}{n\in\omega}
\newcommand{\Hereditary}{\mathcal{H}}
\newcommand{\FatPerf}{\mathit{FatPerf}}
%\renewcommand\abstractname{Summary}
%%%To print the date and time on each page
%%% comment out if not needed (next 14 lines)
\makeatletter
{\newcount\@hour}
{\newcount\@minute}
\def\timenow{\@hour=\time \divide\@hour by 60
\number\@hour:
  \multiply\@hour by 60 \@minute=\time
  \global\advance\@minute by -\@hour
  \ifnum\@minute<10 0\number\@minute\else
  \number\@minute\fi}
\def\ctimenow{\hfil{\tt \jobname.tex, \today~Time: \timenow }\hfil}
      \let\@oddfoot\ctimenow\let\@evenfoot\ctimenow
\makeatother

\pagestyle{myheadings}
\markboth{{\bf (...)}}{\bf \today}

% To see corrections comment next line and uncomment the second one
\newcommand{\correction}[2]{#1}
% \newcommand{\correction}[2]{#2}

\pagestyle{myheadings}
% To see corrections comment next line and uncomment the second one

%\author{}
%\address[]{}

\begin{document}
\title{(...)}

\maketitle

\subsection{Notations and Definitions}
\begin{definition}
\label{definition:a_prime}
$\aidealprime(\tau_1, \tau_2) = \lbrace A \subseteq X\colon \exists_{E \subseteq X} 
X\setminus E \in \tau_1 \wedge A \subseteq E \setminus  \mathint_{\tau_2}(E) \rbrace$
\end{definition}

\subsection{$\aideal$ via Marczewski - Burstin schema}

We may consider the definition of the ideals $\aideal$
as a special kind of general schema via Marczewski - Burstin.
Analogously to the notion of bitopological spaces we call define
a {\it\biMB{}} system:

\begin{definition}
Suppose that $\cF_1$ and $\cF_2$ are collections of nonempty
subset of a set $X$. 
%%%@@TODO: a moze lepiej by brzmialo: analogously like...?
Analogously to the notion of bitopological spaces we call a structure
$\langle \cF_1, \cF_2 \rangle$ a {\it\biMB{}} system.
Notice that we don't require an inclusion $\cF_1 \subseteq \cF_2$.
To avoid trivial cases assume that $\bigcup \cF_1 = \bigcup \cF_2 = X$.
\end{definition}

\begin{definition}
If $\langle \cF_1, \cF_2 \rangle$ is a {\it\biMB{}} system,
then define 
$\aideal(\cF_1, \cF_2) = \lbrace A \subseteq \real\colon 
\forall_{W\in \cF_2} \exists_{U \in \cF_1}
W \cap U \not= \emptyset \wedge W \cap U \cap A = \emptyset
\rbrace$ 
\end{definition}

Notice that such defined collection $\aideal(\cF_1, \cF_2)$
in general case need not be an ideal.
%%%@@TODO: podac tu jakis ciekawy - naturalny - kontrprzyklad.

Nevertheless, let us notice that the collection
$\aideal(\tau_e\setminus\lbrace\emptyset\rbrace, \Perf)$ is an ideal. Namely we have:

\begin{theorem}
$\aideal(\tau_e\setminus\lbrace\emptyset\rbrace, \Perf)$ is an ideal.
\end{theorem}

\begin{proof}
Let $A, B \in \aideal(\tau_e\setminus\lbrace\emptyset\rbrace, \Perf)$,
and let $P$ be a perfect set. 
There exist $W_A \in \tau_e$ such that 
$W_A \cap P \not= \emptyset \wedge W_A \cap P \cap A = \emptyset$.
Let $P_1 = \cl_{\tau_e}(P \cap W_1)$. It is well know that such defined
$P_1$ is also a perfect set. So let $W_B \in \tau_e$ 
be such that 
$W_B \cap P_1 \not= \emptyset \wedge W_B \cap P_1 \cap B = \emptyset$.
Define $W = W_A \cap W_B$. Then we have:
$W \cap P = W_B \cap W_A \cap P$, 
but since $W_B \cap \cl_{\tau_e}(P \cap W_A) \not= \emptyset$
we have $W_B \cap P \cap W_A \not= \emptyset$,
hence $W \cap P \not= \emptyset$.
Moreover, we have
$W \cap P \cap (A \cup B) = (W \cap P \cap A) \cup (W \cap P \cap B) \subseteq$
$(W_A \cap P \cap A) \cup (W_B \cap W_A \cap P \cap B) =$
$W_B \cap W_A \cap P \cap B \subseteq W_B \cap \cl_{\tau_e} (P \cap W_A) \cap B =$
$W_B \cap P_1 \cap B = \emptyset$.
\end{proof}

Let us formulate a general theorem:

\begin{theorem}
Suppose that $\langle \cF_1, \cF_2 \rangle$ is a \biMB{} system.
Assume that 
\begin{enumerate}	
\item
\label{mb-condition-subset}
  $\cF_1 \subseteq \cF_2$;
\item 
\label{mb-condition-f1}
 $\forall_{E,F \in \cF_1} \exists{\cG \subseteq \cF_1} E \cap F 
 = \bigcup \cG$
 (in particular, 
 this condition holds if $\forall_{E,F \in \cF_1} E \cap F \not= \emptyset \implies E \cap F \in \cF_1$);  
\item
\label{mb-condition-f1f2}
  $\forall_{E \in cF_1} \forall_{W \in \cF_2} E \cap W \not= \emptyset \implies E \cap W \in \cF_2$  
\end{enumerate}	
Then the collection
$\aideal(\cF_1, \cF_2) = \lbrace A \subseteq \real\colon 
\forall_{W\in \cF_2} \exists_{U \in \cF_1}
W \cap U \not= \emptyset \wedge W \cap U \cap A = \emptyset
\rbrace$ 
is an ideal. 
\end{theorem}
\begin{proof}
It suffices only to check that if $A, B \in \aideal(\cF_1, \cF_2)$ then
$A\cup B \in \aideal(\cF_1, \cF_2)$.
Suppose that $W \in \cF_2$, then there exists $U\in\cF_1$
such that 
$W \cap U \not= \emptyset$ and $W \cap U \cap A = \emptyset$.
By condition \ref{mb-condition-f1f2} 
we know that $W \cap U \in \cF_2$ so
we conclude that
there exists a set $V \in \cF_1$ such that
$W \cap U \cap V \not= \emptyset$ and $W \cap U \cap V \cap B = \emptyset$.
By condition \ref{mb-condition-f1} we know that 
there exists $V^* \in \cF_1$, $V^* \subseteq U \cap V$ such that 
$W \cap V^* \not= \emptyset$ and $W \cap V^* \cap B = \emptyset$.
Since $W \cap V^* \cap A = \emptyset$ we have that 
$W \cap V^* \cap (A \cup B) = \emptyset$
and this ends the proof.
\end{proof}

\begin{corollary}
$\aideal(\tau_e, \baseEllentuck)$ is an ideal.
\end{corollary} 

\subsection{General $\aidealprime$ collection}

Suppose that $\langle \cF_1, \cF_2 \rangle$ is a \biMB{} system.
Let us define a general $\aidealprime$ collection:

\begin{definition}
A set $A \in \aidealprime(\cF_1, \cF_2)$ iff there exists
a subfamily $\cF^\prime \subseteq \cF_1$ such 
that $A \subseteq X \setminus \big( \bigcup \cF^\prime \cup 
\bigcup \lbrace E \in \cF_2 \colon E \cap \bigcup \cF^\prime = \emptyset \rbrace\big) = \big( X \setminus \bigcup \cF^\prime \big)
\setminus \bigcup \lbrace E \in \cF_2 \colon E \subseteq 
\big( X \setminus \bigcup \cF^\prime \big)\rbrace$.
\end{definition}

  It is easy to see that this definition is indeed a generalization
of the collection $\aidealprime(\tau_1, \tau_2)$ defined in 
Definition \ref{definition:a_prime}. 
By mimic the standard proof about the family $\aideal$
we can prove the following
\begin{fact}
Suppose that $\langle \cF_1, \cF_2 \rangle$ is a \biMB{} system.
Then $\aidealprime(\cF_1, \cF_2) \subseteq \aideal(\cF_1, \cF_2)$.
\end{fact}
\begin{proof}
Suppose that $\cF^\prime \subseteq \cF_1$
and let $W \in \cF_2$. 
Denote $A = \big( X \setminus \bigcup \cF^\prime \big)
\setminus \bigcup \lbrace E \in \cF_2 \colon E \subseteq 
\big( X \setminus \bigcup \cF^\prime \big)\rbrace$
If $W \cap \bigcup \cF^\prime \not= \emptyset$
then $W \cap U \not= \emptyset$ for some 
$U \in \cF^\prime \subseteq \cF_1$.
Then $W \cap U \cap A = \emptyset$.
On the other hand, if $W \cap \bigcup \cF^\prime = \emptyset$
then $W \subseteq \big( X \setminus \bigcup \cF^\prime \big)$
so $W \cap A \not= \emptyset$ and since 
$\bigcup \cF_1 = X$ we can find $U \in \cF_1$
such that $U \cap W \not= \emptyset$.
Hence again $W \cap U \cap A = \emptyset$.
\end{proof}

%%%@@TODO: tutaj ewentualny przyklad z tau_e , perf:


%%%%%%%%%%%%%%%%%%%%%%%%%%%%%%%%%%%%%%%%%%%%%%%%%%%%%%%%%%%%%%%%%%%%%%%%%%%%%
%                        The Bibliography                                   %
%%%%%%%%%%%%%%%%%%%%%%%%%%%%%%%%%%%%%%%%%%%%%%%%%%%%%%%%%%%%%%%%%%%%%%%%%%%%%
\begin{thebibliography}{10}

\bibitem[FG]{FG}
M. Frankowska, S.G\l{}\c{a}b, 
\emph{On some $\sigma$-ideal without ccc}, Colloquium Mathematicum \textbf{158 (1)} (2019), 127--140.
\end{thebibliography}

\end{document}

